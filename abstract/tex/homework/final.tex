\documentclass{article}
\usepackage{neb-titles}
\usepackage{amssymb}

\begin{document}

\TestTitle[class={Abstract I}, name={Final Exam}, term=Fall, year=2015, date={}]

\begin{enumerate}
\item Let $R$ be a ring, with $S \subseteq R$ a subring and $I \subseteq R$ an ideal.

\begin{enumerate}
\item Show that $S \cap I$ is an ideal in $S$.

\item Show that $I$ is an ideal in $S + I$.

\item Show that $S/(S \cap I) \cong (S + I)/I$. (Hint: show that the map $\varphi : S \rightarrow (S+I)/I$ is surjective with kernel $S \cap I$ and use the First Isomorphism Theorem.)
\end{enumerate}

\item Let $R$ be a ring. An element $x \in R$ is called \emph{nilpotent} if $x^n = 0$ for some power $n$. For example, $\overline{2}$ is nilpotent in $\mathbb{Z}/(8)$ since $\overline{2}^3 = 0$.

Show that if $R$ is commutative then the set $N \subseteq R$ consisting of all the nilpotent elements is an ideal.

\item A ring element $x$ is called \emph{idempotent} if $x^2 = x$. For example, 0 is idempotent in any ring since $0^2 = 0$.
\begin{enumerate}
\item Determine which elements of $\mathbb{Z}/(30)$ are idempotent.
\item Determine which elements of $\mathbb{F}_3[x]/(x^2-x)$ are idempotent.
\end{enumerate}
\end{enumerate}

\end{document}
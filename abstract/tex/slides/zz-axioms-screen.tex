\documentclass[handout]{beamer}
\usetheme{Warsaw}
\usecolortheme{default}
\usepackage{amssymb}
\usepackage{amsthm}

\newtheorem{thm}{Theorem}
\newtheorem{prop}[thm]{Proposition}

\begin{document}

\title{Abstract Algebra \\ Day 1: The $\mathbb{Z}$ Axioms}
\author{Nathan Bloomfield}

\begin{frame}
\titlepage
\end{frame}



\begin{frame}
\frametitle{The $\mathbb{Z}$ Axioms}
There is a set $\mathbb{Z}$, whose elements are called \emph{integers}, which is equipped with two operations $+$ and $\cdot$ and a binary relation $\leq$ which satisfy the following properties.
\end{frame}



\begin{frame}
\frametitle{The $\mathbb{Z}$ Axioms: Arithmetic}
\begin{itemize}
\item[A1.] $a+(b+c) = (a+b)+c$ \textcolor{lightgray}{for all $a,b,c \in \mathbb{Z}$.} \pause
\item[A2.] \textcolor{lightgray}{There is an integer 0 such that} $a + 0 = 0 + a = a$ \textcolor{lightgray}{for all $a \in \mathbb{Z}$.} \pause
\item[A3.] \textcolor{lightgray}{For every $a \in \mathbb{Z}$ there is a unique integer, denoted $-a$, such that} $a + (-a) = (-a) + a = 0$. \pause
\item[A4.] $a + b = b + a$ \textcolor{lightgray}{for all $a,b \in \mathbb{Z}$.} \pause
\item[M.] $a(bc) = (ab)c$ \textcolor{lightgray}{for all $a,b,c \in \mathbb{Z}$.} \pause
\item[D.] $a(b+c) = ab + ac$ and $(b+c)a = ba + ca$ \textcolor{lightgray}{for all $a,b,c \in \mathbb{Z}$.} \pause
\item[C.] $ab = ba$ \textcolor{lightgray}{for all $a,b \in \mathbb{Z}$.} \pause
\item[U.] \textcolor{lightgray}{There is an integer 1 such that} $a \cdot 1 = 1 \cdot a = a$ \textcolor{lightgray}{for all $a \in \mathbb{Z}$.} \pause
\item[Z.] If $ab = 0$, then either $a = 0$ or $b = 0$ \textcolor{lightgray}{for all $a,b \in \mathbb{Z}$.}
\end{itemize}
\end{frame}



\begin{frame}
\frametitle{The $\mathbb{Z}$ Axioms: Order}
\begin{itemize}
\item[P1.] $a \leq a$ \textcolor{lightgray}{for all $a \in \mathbb{Z}$.} \pause
\item[P2.] If $a \leq b$ and $b \leq a$ then $a = b$ \textcolor{lightgray}{for all $a,b \in \mathbb{Z}$.} \pause
\item[P3.] If $a \leq b$ and $b \leq c$ then $a \leq c$ \textcolor{lightgray}{for all $a,b,c \in \mathbb{Z}$.} \pause
\item[P4.] Either $a \leq b$ or $b \leq a$ \textcolor{lightgray}{for all $a,b \in \mathbb{Z}$.} \pause
\item[O1.] If $a \leq b$ then $a+c \leq b+c$ \textcolor{lightgray}{for all $a,b,c \in \mathbb{Z}$.} \pause
\item[O2.] If $0 \leq a$ and $0 \leq b$ then $0 \leq ab$ \textcolor{lightgray}{for all $a,b \in \mathbb{Z}$.} \pause
\item[O3.] $0 < 1$.
\end{itemize}
\end{frame}



\begin{frame}
\frametitle{The $\mathbb{Z}$ Axioms: Well-Ordering Property}

We call \[\mathbb{N} = \{a \in \mathbb{Z} \mid 0 \leq a\}\] the set of \emph{natural numbers}. \pause

\vspace{1cm}

\begin{itemize}
\item[WOP.] Every nonempty subset of $\mathbb{N}$ has a $\leq$-least element. \pause
\end{itemize}

\vspace{1cm}

That is, if $S \subseteq \mathbb{N}$ is not empty, there is a natural number $m \in S$ such that $m \leq s$ for all $s \in S$.
\end{frame}



\begin{frame}
\frametitle{Consequences}

These 16 axioms uniquely characterize the ``integers'' we know and love; any other provably true statement about $\mathbb{Z}$ can be derived from them. For example: \pause

\begin{itemize}
\item $(-1) \cdot a = -a$ for all $a \in \mathbb{Z}$. \pause
\item $a \cdot 0 = 0 \cdot a = 0$ for all $a \in \mathbb{Z}$. \pause (Use D) \pause
\item If $a \leq b$ and $0 \leq c$, then $ac \leq bc$. \pause (Use D) \pause
\item There is no integer $t$ such that $0 < t < 1$. \pause (Use WOP; $t^2 < t$) \pause
\item If $a < b$ then $a+1 \leq b$ \pause
\item Exactly one of $a < 0$, $a = 0$, and $a > 0$ is true. \pause
\item Every element of $\mathbb{N}$ is either 0 or of the form $n+1$ where $n \in \mathbb{N}$. \pause
\item ... etc.
\end{itemize}
\end{frame}



\begin{frame}
\frametitle{Principle of Mathematical Induction}

\begin{thm}[Induction]
Suppose $B \subseteq \mathbb{N}$ is a subset such that
\begin{itemize}
\item $0 \in B$ (the \emph{Base Case}) and
\item If $n \in B$, then $n + 1 \in B$ (the \emph{Inductive Step}).
\end{itemize}
Then $B = \mathbb{N}$.
\end{thm}

\begin{thm}[Strong Induction]
Suppose $B \subseteq \mathbb{N}$ is a subset such that
\begin{itemize}
\item $0 \in B$ and
\item If $k \in B$ for all $0 \leq k \leq n$, then $n+1 \in B$.
\end{itemize}
Then $B = \mathbb{N}$.
\end{thm}

Proof: Use WOP. These two statements are equivalent in power, but sometimes Strong Induction is convenient.
\end{frame}



\begin{frame}
\frametitle{Principle of Mathematical Induction: Examples}

\begin{prop}
For all natural numbers $n$, we have \[ \sum_{i=1}^n i = \frac{n(n+1)}{2}. \]
\end{prop} \pause

\begin{prop}
For all natural numbers $n$, we have \[ \sum_{i=1}^n \frac{1}{i(i+1)} = \frac{n}{n+1} \] (Hint: Use two base cases, 0 and 1.)
\end{prop}
\end{frame}



\end{document}

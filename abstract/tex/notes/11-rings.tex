\documentclass{article}
\usepackage{neb-macros}

\begin{document}

\CheapTitle{Rings}

\begin{dfn}[Ring]
A \emph{ring} is a set $R$ equipped with two operations $+$ and $\cdot$, which satisfy the following properties.
\begin{itemize}
\item[A1.] $(a+b)+c = a+(b+c)$ for all $a,b,c \in R$.
\item[A2.] There is an element $0_R \in R$ (called a \emph{zero}) such that $a+0_R = 0_R+a = a$ for all $a \in R$.
\item[A3.] For every $a \in R$ there is an element $-a \in R$ (called a \emph{negative} of $a$) such that $a+(-a) = (-a)+a = 0_R$.
\item[A4.] $a + b = b + a$ for all $a,b \in R$.
\item[M.] $(ab)c = a(bc)$ for all $a,b,c \in R$.
\item[D.] $a(b+c) = ab + ac$ and $(b+c)a = ba + ca$ for all $a,b,c \in R$.
\end{itemize}
\end{dfn}

\begin{prop}
Let $R$ be a ring.
\begin{enumerate}
\item The zero element of $R$ is unique in the following sense: if $a,b \in R$ such that $a+b = a$, then $b = 0_R$.
\item Negative elements in $R$ are unique in the following sense: If $a,b \in R$ such that $a+b = 0_R$, then $b = -a$.
\item $-(-a) = a$ for all $a \in R$.
\item $0_R \cdot a = a \cdot 0_R = 0_R$ for all $a \in R$.
\item $(-a)b = a(-b) = -(ab)$ for all $a,b \in R$.
\item $(-a)(-b) = ab$ for all $a,b \in R$.
\end{enumerate}
\end{prop}

\begin{proof} \mbox{}
\begin{enumerate}
\item Suppose $a+b = a$. Now $-a + (a+b) = -a + a$, and by A1 $(-a + a) + b = -a + a$. By A3 we have $0_R + b = 0_R$, and by A2 we have $b = 0_R$.
\item Suppose $a + b = 0_R$. Now $-a + (a+b) = -a + 0_R$, and by A1 we have $(-a+a)+b = -a+0_R$. By A3 we have $0_R + b = -a + 0_R$, and using A2 (twice) we have $b = -a$.
\item By definition, $(-a) + a = 0_R$, so by the uniqueness of negatives we have $a = -(-a)$.
\item Let $a \in R$. Now $a \cdot a + 0_R \cdot a = (a + 0_R) \cdot a = a \cdot a$, and so $0_R \cdot a = 0_R$. The other equality is similar.
\item Let $a,b \in R$. Now $(-a)b + ab = (-a + a)b = 0_R \cdot b = 0_R$, so that $(-a)b = -(ab)$. The other equality is similar.
\item Using the previous statement, we have $(-a)(-b) = -(a(-b)) = -(-(ab)) = ab$.
\end{enumerate}
\end{proof}

\subsection*{Examples}

\begin{itemize}
\item[$\Ints, \Ints/(n)$] The integers are a ring by definition, and we showed that the integers mod $n$ are a ring for any $n > 0$.

\item[$\QQ$] The rational numbers are a ring under the usual addition and multiplication; we will prove this later. (Actually we will define the rational numbers.)

\item[$\RR$, $\CC$] The real and complex numbers are also rings, although even defining these sets of ``numbers'' is beyond the scope of this text.

\item[$0$] The smallest possible ring must have at least one element, the zero. Suppose this is \emph{all} we have. Now the arithmetic is pretty boring: $0+0 = 0$ and $0 \cdot 0 = 0$. It is straightforward to check that these operations make the set $\{0\}$ into a ring. This example isn't very interesting, so we call this the \emph{trivial ring} or the \emph{zero ring}.

\item[$R^A$] Let $R$ be a ring, and let $A$ be any nonempty set. Then the set \[ R^A = \{ \varphi \mid \varphi : A \rightarrow R \} \] is a ring under the ``pointwise'' operations \[ (\alpha + \beta)(x) = \alpha(x) + \beta(x) \quad \mathrm{and} \quad (\alpha\beta)(x) = \alpha(x) \beta(x). \]

\item[$\MAT{2}{R}$] Let $R$ be a ring, and consider the set \[ \MAT{2}{R} = \left\{ \begin{bmatrix} a_{11} & a_{12} \\ a_{21} & a_{22} \end{bmatrix} \mid a_{11}, a_{12}, a_{21}, a_{22} \in R \right\}. \] These are just the $2 \times 2$ matrices with entries in $R$. The usual matrix addition and multiplication make $\MAT{2}{R}$ into a ring. Specifically, we define

\[\begin{bmatrix} a_{11} & a_{12} \\ a_{21} & a_{22} \end{bmatrix} + \begin{bmatrix} b_{11} & b_{12} \\ b_{21} & b_{22} \end{bmatrix} = \begin{bmatrix} a_{11} + b_{11} & a_{12} + b_{12} \\ a_{21} + b_{21} & a_{22} + b_{22} \end{bmatrix}\]

and

\[\begin{bmatrix} a_{11} & a_{12} \\ a_{21} & a_{22} \end{bmatrix} \cdot \begin{bmatrix} b_{11} & b_{12} \\ b_{21} & b_{22} \end{bmatrix} = \begin{bmatrix} a_{11}b_{11} + a_{12}b_{21} & a_{11}b_{12} + a_{12}b_{22} \\ a_{21}b_{11} + a_{22}b_{21} & a_{21}b_{12} + a_{22}b_{22} \end{bmatrix}.\]

\item[$2\Ints$] Consider the set \[ 2\Ints = \{ 2k \mid k \in \Ints \}. \] It is not too difficult to show that this set is a ring under the usual addition and multiplication of integers.

\item[$2^X$] Let $X$ be any nonempty set. The powerset $2^X$ is a ring under the operations $A + B = (A \setminus B) \cup (B \setminus A)$ and $A \cdot B = A \cap B$. This is called a \emph{ring of sets}.
\end{itemize}

\subsection*{Commutative and Unital Rings}

Our list of examples is starting to get complicated, so we make two additional definitions to start drawing distinctions among them.

\begin{dfn}
Let $R$ be a ring.
\begin{itemize}
\item We say that $R$ is \emph{commutative} if it satisfies the following additional property.
\begin{itemize}
\item[C.] $ab = ba$ for all $a,b \in R$.
\end{itemize}
\item We say that $R$ is \emph{unital} if it satisfies the following additional property.
\begin{itemize}
\item[U.] There is an element $1 \in R$ (called a \emph{one}) such that $1 \cdot a = a \cdot 1 = a$ for all $a \in R$.
\end{itemize}
\end{itemize}
\end{dfn}

\begin{prop}
Let $R$ be a unital ring.
\begin{enumerate}
\item The one element of $R$ is unique in the following sense: if $u \in R$ such that $u \cdot a = a$ for all $a \in R$, then $u = 1$.
\item $-a = (-1) \cdot a$ for all $a \in R$.
\end{enumerate}
\end{prop}

\begin{proof}
\begin{enumerate}
\item Suppose $u$ is such an element. In particular, $1 = u \cdot 1 = u$.
\item Let $a \in R$. Then $a + (-1)a = 1a + (-1)a = (1 + (-1))a = 0a = 0$, so that $(-1)a = -a$.
\end{enumerate}
\end{proof}

\end{document}

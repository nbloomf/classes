\documentclass{article}
\usepackage{neb-macros}
\usepackage{tikz}
  \usetikzlibrary{patterns}

\begin{document}

\CheapTitle{Over a GCD Domain -- Part I}

One of our big questions is to what extent the structure of $R$ is reflected in the structure of $R[x]$; if $R$ has more ``technology'' available, perhaps this can be used to say interesting things about the polynomials over $R$. In this section we will see that this is indeed the case if $R$ is a GCD domain.

In fact, thanks to the polynomial long division algorithm, if $R$ is a domain then $R[x]$ is already sitting inside a Euclidean domain -- namely $F[x]$ where $F$ is the field of fractions of $R$. So it doesn't take much to get extra technology in $R[x]$.

\begin{dfn}[Content of a polynomial]
Let $R$ be a GCD domain and let $p(x) \in R[x]$ be a polynomial with coefficients $a_i$. We define the \emph{content} of $p(x)$ to be \[ \content{p} = \left\{ \begin{array}{ll} 0 & \mathrm{if}\ p(x) = 0 \\ \GCDS{a_0,a_1,\ldots,a_d} & \mathrm{if}\ p(x) \neq 0, \mathrm{where}\ d = \deg{p}. \end{array} \right. \] If $\content{p} = 1$, we say that $p(x)$ is \emph{primitive}.
\end{dfn}

For example, $\ZZ$ is a GCD Domain, and $\content{2x^3 + 4x - 6} = 2$. Every field $F$ is a GCD domain, and every nonzero polynomial over $F$ is primitive.

\begin{prop}
Let $R$ be a GCD domain.
\begin{enumerate}
\item Every polynomial $a(x) \in R[x]$ can be written as $a(x) = \content{a} \overline{a}(x)$, where $\overline{a}(x) \in R[x]$ is primitive.
\item Let $d \in R$ and $a(x) \in R[x]$. Then the constant polynomial $d$ divides $a(x)$ in $R[x]$ if and only if $d$ divides $\content{a}$ in $R$.
\item Let $d \in R$ and $a(x), b(x) \in R[x]$. If $d|\content{a+b}$ and $d|\content{a}$, then $d|\content{b}$.
\item If $d \in R$ and $a(x) \in R[x]$, then $\content{da} = d\content{a}$.
\item Let $F$ be the field of fractions of $R$ and let $q(x) \in F[x]$. Then there is a fraction $\frac{u}{v} \in F$ such that $p(x) = \frac{u}{v}q(x)$ is in $R[x]$ and is primitive there.

\item $\content{x^n a(x)} = \content{a(x)}$.
\end{enumerate}
\end{prop}

\begin{proof} \mbox{}
\begin{enumerate}
\item If $a(x) = 0$, set $\overline{a}(x) = 1$. Suppose $a(x) \neq 0$. Now $\content{a} = \GCDS{a_0,a_1,\ldots,a_n}$, and in particular for each $i$ we have $a_i = \content{a}\overline{a}_i$ for some $\overline{a}_i$, and $\GCDS{\overline{a}_0, \overline{a}_1, \ldots, \overline{a}_n} = 1$. Let $\overline{a}(x) = \sum_{i=0}^n \overline{a}_i x^i$.

\item (write these)
\end{enumerate}
\end{proof}

\begin{prop}[Gauss' Lemma -- Part I]
Let $R$ be a GCD Domain with $a(x), b(x) \in R[x]$. Then we have the following.
\begin{enumerate}
\item If $a(x)$ and $b(x)$ are primitive, then $a(x)b(x)$ is primitive.
\item $\content{ab} = \content{a}\content{b}$
\item If $a(x)|b(x)$ in $R[x]$, then $\content{a}|\content{b}$ in $R$.
\end{enumerate}
\end{prop}

\begin{proof} \mbox{}
\begin{enumerate}
\item We proceed by induction on the number $k$ of nonzero terms of $a$ and $b$ together.
\begin{enumerate}
\item \textbf{Base Case} ($k = 0$): If $a$ and $b$ together have no nonzero terms, then $a(x) = b(x) = 0$; neither is primitive.
\item \textbf{Base Case} ($k = 1$): If $a$ and $b$ together have exactly one nonzero term, then either $a(x) = 0$ or $b(x) = 0$; one is not primitive.
\item \textbf{Base Case} ($k = 2$): If $a(x)$ and $b(x)$ together have exactly two nonzero terms, then each must have exactly one. (Otherwise one is zero and thus not primitive.) Say $a(x) = a_n x^n$ and $b(x) = b_m x^m$. If both $a(x)$ and $b(x)$ are primitive, then $a_n = \content{a}$ and $b_m = \content{b}$ are units, so that $\content{ab} = a_nb_m$ is a unit; hence $a(x)b(x)$ is primitive.
\item \textbf{Inductive Step:} Suppose the result holds for all pairs of primitive polynomials having less than $n > 2$ nonzero terms together, and suppose that $a(x)$ and $b(x)$ are primitive with exactly $n$ nonzero terms together. Say $\deg{a} = n$ and $\deg{b} = m$, so that the leading coefficients of $a$, $b$, and $ab$ are $a_n$, $b_m$, and $a_nb_m$, respectively. Now let $c = \content{ab}$, and suppose BWOC that $c$ is \emph{not} a unit. Note that $c|a_nb_m$. Now $\GCD{c}{a_n}$ and $\GCD{c}{b_m}$ cannot both be units in $R$. (If $\GCD{c}{a_n} = 1$, then by Euclid's lemma we have $c | \GCD{c}{b_m}$.) So suppose WLOG that $\GCD{c}{a_n} = d$ is not a unit.

Now $d|\content{ab}$ in $R$, so that $d|a(x)b(x)$ in $R[x]$. Since $d|a_n$, we also have $d|a_nx^n$ in $R[x]$. Thus $d|b(x)(a(x) - a_nx^n)$ in $R[x]$, and thus \[ d | \content{b(x)(a(x) - a_nx^n)} = \content{a(x) - a_nx^n}\content{b(x)p(x)}, \] where $p(x) \in R[x]$ is primitive such that $a(x) - a_nx^n = \content{a(x) - a_nx^n}p(x)$. In particular, note that $p(x)$ and $a(x) - a_nx^n$ have the same number of nonzero terms which is one fewer than the number of nonzero terms of $a(x)$. Thus $b$ and $p$ have fewer than $n$ nonzero terms. Since $b$ and $p$ are both primitive, by the inductive hypothesis, $\content{bp} = 1$. Thus we have $d|\content{a(x) - a_nx^n}$. Since $d|\content{a_nx^n}$, by the lemma we have $d|\content{a}$. But $a$ is primitive, so that $d$ is a unit, a contradiction. So $a(x)b(x)$ must be primitive.
\end{enumerate}

\item We have
\begin{eqnarray*}
\content{a(x)b(x)} & = & \content{\content{a}\overline{a}\content{b}\overline{b}} \\
 & = & \content{a}\content{b}\content{\overline{a}\overline{b}} \\
 & = & \content{a}\content{b}
\end{eqnarray*}

\item Say $a(x)c(x) = b(x)$; then $\content{a}\content{c} = \content{b}$. \qedhere
\end{enumerate}
\end{proof}

\begin{lem}
Let $R$ be a GCD domain with field of fractions $F$.
\begin{enumerate}
\item If $p(x) \in R[x]$ is primitive, $r \in R$, and $a(x) \in R$ such that $p(x)|a(x)$ and $r|a(x)$ in $R[x]$, then $rp(x)|a(x)$ in $R[x]$.
\item If $q(x) \in F[x]$ and $p(x) \in R[x]$ such that $p(x)$ is primitive and $p(x)q(x) \in R[x]$, then in fact $q(x) \in R[x]$.
\end{enumerate}
\end{lem}

\begin{proof} \mbox{}
\begin{enumerate}
\item Write $a(x) = p(x)b(x)$ with $b(x) \in R[x]$. Since $r|a(x)$, we have $r|\content{a} = \content{p}\content{b} = \content{b}$, since $p$ is primitive. So $r|b(x)$ in $R[x]$. Say $b(x) = rc(x)$; then $a(x) = rp(x)c(x)$ as needed.

\item We have $\frac{u}{v} \in F$ (in lowest terms) such that $\frac{u}{v}q(x) \in R[x]$ is primitive; say $\frac{u}{v}q(x) = s(x)$. Now $uq(x) = vs(x)$, and moreover $up(x)q(x) = vp(x)s(x) \in R[x]$. Now
\begin{eqnarray*}
u \cdot \content{pq} & = & \content{up(x)q(x)} \\
 & = & \content{vp(x)s(x)} \\
 & = & v \cdot \content{ps} = v,
\end{eqnarray*}
since $p$ and $s$ are primitive in $R[x]$. In particular, $u|v$. Since $\frac{u}{v}$ is in lowest terms, without loss of generality, $u = 1$, so that $\frac{1}{v}q(x) = s(x)$. Thus $q(x) = vs(x) \in R[x]$ as needed. \qedhere
\end{enumerate}
\end{proof}

\begin{prop}[Gilmer-Parker]
If $R$ is a GCD Domain, then $R[x]$ is a GCD Domain.
\end{prop}

\begin{proof}
Let $a(x), b(x) \in R[x]$. Let $k = \GCD{\content{a}}{\content{b}}$ (remember that $R$ is a GCD domain). Let $F$ be the field of fractions of $R$. Now $F[x]$ is a Euclidean domain, in particular a GCD domain, so that $a(x)$ and $b(x)$ have a greatest common divisor in $F[x]$. By the lemma, we can take an associate (in $F[x]$) of this gcd which is in $R[x]$ and primitive; say $t(x)$. We claim that $kt(x)$ is a gcd of $a$ and $b$ in $R[x]$.

First note that $k|\content{a}$, so that $k|a{x}$. Now $t(x)|a(x)$ in $F[x]$, where $t$ and $a$ are in $R[x]$ and $t(x)$ is primitive. By the lemma, $t(x)|a(x)$ in $R[x]$, and again using the lemma, $kt(x)|a(x)$ in $R[x]$. Similarly, $kt(x)|b(x)$ in $R[x]$. So $kt(x)$ is a common divisor of $a(x)$ and $b(x)$ in $R[x]$.

Now suppose that $e(x) \in R[x]$ is a common divisor of $a(x)$ and $b(x)$ over $R$. If $e(x)$ is constant, then $e(x) = e_0 | \GCD{\content{a}, \content{b}} = k$. Suppose instead that $e(x)$ has positive degree. Now $e(x)$ divides $a(x)$ and $b(x)$ in $F[x]$, which is a GCD domain, and thus $e(x)$ divides $t(x)$ in $F[x]$. Say $e(x)f(x) = t(x)$ where $f(x) \in F[x]$. By the lemma, we may write $f(x) = \frac{u}{v}g(x)$ where $g(x) \in R[x]$ is primitive and $\GCD{u}{v} = 1$. We have $ue(x)g(x) = vf(x) \in R[x]$. Now $\content{ue(x)g(x)} = \content{vt(x)}$, and since $g$ and $t$ are primitive over $R$, $u\content{e} = v$. By Euclid's lemma, $v|\content{e}$, so that $v|\content{a}$ and $v|\content{b}$, and thus $v|k$. In particular, we have $kf(x) = k\frac{u}{v}g(x) \in R[x]$, and thus $e(x) \cdot kf(x) = kt(x)$, so that $e(x)|kt(x)$ in $R[x]$.

Thus $kt(x)$ is a greatest common divisor of $a(x)$ and $b(x)$ in $R[x]$.
\end{proof}

\subsection*{Exercises}

\begin{enumerate}
\item Let $R$ be a GCD domain with $p(x), q(x) \in R[x]$ so that $q$ is irreducible (hence prime), and let $k$ be a natural number. Show that $q^{k+1}$ divides $p$ in $R[x]$ iff $q|p$ and $q^k|p'$ in $R[x]$. In particular, show that $p$ is squarefree iff $\GCD{p}{p'} = 1$.

\item Let $R$ be a GCD domain, with $p,q \in R[x]$ nonzero. Show that $p$ and $q$ have a common factor of positive degree in $R[x]$ if and only if there exist $a,b \in R[x]$, not zero, such that $\deg{a} < \deg{q}$, $\deg{b} < \deg{p}$, and $pa - qb = 0$. (Looking forward to univariate resultant.)
\end{enumerate}

\end{document}

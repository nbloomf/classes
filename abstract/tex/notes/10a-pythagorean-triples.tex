\documentclass{article}
\usepackage{neb-macros}
\usepackage{tikz}
  \usetikzlibrary{patterns}

\begin{document}

\CheapTitle{Pythagorean Triples}

\begin{dfn}[Pythagorean Triple]
Given positive integers $a$, $b$, and $c$, we say that $(a,b,c)$ is a \emph{pythagorean triple} if $a^2 + b^2 = c^2$. If $a$, $b$, and $c$ are mutually coprime we say that as a pythagorean triple, $(a,b,c)$ is \emph{primitive}.
\end{dfn}

\begin{prop}
Let $(a,b,c)$ be a pythagorean triple.
\begin{enumerate}
\item The following are equivalent. (1) $(a,b,c)$ is primitive, (2) $\GCD{a}{b} = 1$, (3) $\GCD{a}{c} = 1$, (4) $\GCD{b}{c} = 1$.
\item If $(a,b,c)$ is primitive, then up to a swap of $a$ and $b$, we can assume that $b$ is even and $a$ and $c$ are odd.
\end{enumerate}
\end{prop}

\begin{proof}
Note that the squares modulo 4 are 0 and 1; considering the equation $a^2 + b^2 \equiv c^2 \mod 4$ then gives only two possibilities: either $a^2 \equiv c^2 \equiv 1 \mod 4$ and $b^2 \equiv 0 \mod 4$, or $b^2 \equiv c^2 \equiv 1 \mod 4$ and $a^2 \equiv 0 \mod 4$. Without loss of generality, we can suppose the first case. Now $b$ is even and $a$ and $c$ odd.
\end{proof}

\begin{lem} \mbox{}
\begin{enumerate}
\item If $a,b,c \in \ZZ$ such that $ab = c^2$ and $\GCD{a}{b} = 1$, then $a = u^2$ and $b = v^2$ are squares.
\item If $a,b \in \ZZ$ are positive and $a^2 = b^2$, then $a = b$.
\end{enumerate}
\end{lem}

\begin{proof} \mbox{}
\begin{enumerate}
\item We can induct on the number of prime factors of $a$. If $a = 1$, then $a = 1^2$ and $b = c^2$. Now suppose $p$ is a prime with $p|a$. Now $p|c^2$, so that $p|c$ (since $p$ is prime) and thus $p^2|c^2$. So $p^2|ab$, and since $p$ does not divide $b$, using Euclid's lemma we have $p^2|a$. Dividing out $p^2$ we get a similar equation $(a')b = (c')^2$ in which $a$ has two fewer prime factors.
\item We have $(a+b)(a-b) = 0$, so either $a = -b$ or $a = b$. In the first case, $a$ is both positive and negative, a contradiction. \qedhere
\end{enumerate}
\end{proof}

\begin{thm}[Euclid's Parameterization of Pythagorean Triples]
Let $a,b,c \in \ZZ$. Then $(a,b,c)$ is a primitive pythagorean triple with $b$ even if and only if there exist integers $m$ and $n$ such that the following hold.
\begin{itemize}
\item $m > n > 0$,
\item $\GCD{m}{n} = 1$,
\item $m-n$ is odd, and
\item $a = m^2 - n^2$, $b = 2mn$, and $c = m^2 + n^2$.
\end{itemize}
\end{thm}

\begin{proof}
First suppose that $m$ and $n$ have these four properties. Certainly $a$, $b$, and $c$ are positive, $b$ is even, and
\begin{eqnarray*}
a^2 + b^2 & = & (m^2 - n^2)^2 + (2mn)^2 \\
 & = & m^4 - 2m^2n^2 + n^4 + 4m^2n^2 \\
 & = & m^4 + 2m^2n^2 + n^4 \\
 & = & (m^2 + n^2)^2 \\
 & = & c^2,
\end{eqnarray*}
so that $(a,b,c)$ is a pythagorean triple. It remains to be seen that $(a,b,c)$ is primitive. To this end, suppose $p$ is a prime dividing both $a = m^2 - n^2 = (m+n)(m-n)$ and $b = 2mn$. If $p = 2$, then 2 divides either $m+n$ or $m-n$. But $m+n \equiv m-n \equiv 1 \mod 2$, a contradiction. If $p \neq 2$, then either $p|m$ or $p|n$ and either $p|(m+n)$ or $p|(m-n)$. If $p|m$ and $p|(m+n)$, then $p|n$, so that $p|\GCD{m}{n}$, a contradiction; similarly, in the other three cases we get a prime divisor of $\GCD{m}{n}$. So in fact $\GCD{a}{b} = 1$, and thus $(a,b,c)$ is a primitive pythagorean triple.

Conversely, suppose $(a,b,c)$ is a primitive pythagorean triple with $b$ even and $a$ and $c$ odd. Note that $c+a$ and $c-a$ are even (consider these equations mod 2). Let's write \[ c+a = 2r, \quad c-a = 2s, \quad \mathrm{and}\ quad b = 2t. \] Now we have $b^2 = c^2 - a^2 = (c+a)(c-a)$, so that $t^2 = rs$.

We claim that $\GCD{r}{s} = 1$. To see this, suppose $p$ is a prime such that $p|r$ and $p|s$. In particular, $p$ divides $c+a$ and $c-a$, so $p$ divides both $2a =(c+a) - (c-a)$ and $2c = (c+a) + (c-a)$. If $p \neq 2$, then $p|\GCD{a}{c}$, so that $p = 1$, a contradiction. Suppose $p = 2$. In this case we have that $c+a = 4r'$ and $c-a = 4s'$, so that $2c = 4(r'+s')$ and $2a = 4(r'-s')$, and thus $2|\GCD{a}{c}$, again a contradiction. So $\GCD{r}{s} = 1$.

Since $rs = t^2$ and $\GCD{r}{s} = 1$, both $r = m^2$ and $s = n^2$ are squares by the lemma. We can assume that $m$ and $n$ are both positive. Since $a$ is positive, we have $m > n$. We can see that $a = m^2 - n^2$ and $c = m^2 + n^2$, and $b^2 = (2mn)^2$, so that $b = 2mn$ by the lemma. Since $\GCD{r}{s} = 1$, we also have $\GCD{m}{n} = 1$. Finally, if $m-n$ is even, then $a^2 = (m-n)(m+n)$ is even, so that $a$ is even, a contradiction; hence $m-n$ is odd.
\end{proof}

\end{document}

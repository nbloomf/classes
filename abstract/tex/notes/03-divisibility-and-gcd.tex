\documentclass{article}
\usepackage{neb-macros}

\begin{document}

\CheapTitle{Divisibility and GCD}

\begin{dfn}[Divides]
Given integers $a$ and $b$, we say that $a$ \emph{divides} $b$, written $a|b$, if there is an integer $c$ such that $ac = b$. In this case we say that $a$ is a \emph{divisor} of $b$.
\end{dfn}

\begin{prop} \mbox{}
\begin{itemize}
\item $a|0$ for all integers $a$.
\item $1|a$ for all integers $a$.
\item $a|a$ for all integers $a$.
\item If $a|b$, then $(-a)|b$ and $a|(-b)$.
\item If $a|b$ and $b \neq 0$, then $0 < |a| \leq |b|$.
\end{itemize}
\end{prop}

\begin{dfn}
Let $a$ and $b$ be integers.
\begin{itemize}
\item We say that an integer $c$ is a \emph{common divisor} of $a$ and $b$ if $c|a$ and $c|b$.
\item We say that an integer $d$ is a \emph{greatest common divisor} of $a$ and $b$ if $d$ is a common divisor, and if $c$ is another common divisor, then $c \leq d$.
\end{itemize}
\end{dfn}

\begin{prop}
Any two integers (not both zero) have a unique greatest common divisor, which we denote $\GCD{a}{b}$. We also define $\GCD{0}{0} = 0$ as a special case.
\end{prop}

\begin{prop} \mbox{}
\begin{itemize}
\item $\GCD{a}{b} = \GCD{b}{a}$ for all integers $a$ and $b$.
\item $\GCD{a}{a} = |a|$ for all integers $a$.
\item If $a$ and $b$ are integers with $b|a$, then $\GCD{a}{b} = |b|$.
\item $\GCD{a}{1} = 1$ for all integers $a$.
\item $\GCD{a}{0} = |a|$ for all integers $a$.
\end{itemize}
\end{prop}

\begin{prop}[Euclidean Algorithm]
If $a$ and $b$ are integers with $b > 0$, and if $a = qb + r$ where $0 \leq r < b$, then $\GCD{a}{b} = \GCD{b}{r}$.
\end{prop}

\begin{proof}
Let $d = \GCD{a}{b}$ and $e = \GCD{b}{r}$. We need to show that $d = e$; to do this, we will show that $d \leq e$ and $e \leq d$.
\begin{itemize}
\item By definition we have $d|a$ and $d|b$; that is, $a = da'$ and $b = db'$ for some integers $a'$ and $b'$. Now \[ r = a - qb = da' - qdb' = d(a' - qb'), \] so that $d|r$. In particular, $d$ is a common divisor of $b$ and $r$, and so $d \leq e$.
\item Similarly, we have $e|b$ and $e|r$, so that $e|a$, and thus $e \leq d$. \qedhere
\end{itemize}
\end{proof}

The Euclidean Algorithm gives us a way to explicitly compute the GCD of two integers \emph{as long as} we can compute quotients and remainders as in the Division Algorithm; in fact, it is quite fast. Note that since $r$ is strictly less than $b$, this recursion must eventually terminate with a statement of the form $\GCD{a}{0}$.

\end{document}

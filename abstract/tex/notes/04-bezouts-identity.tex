\documentclass{article}
\usepackage{neb-macros}

\begin{document}

\CheapTitle{Bezout's Identity}

\begin{thm}[Bezout's Identity]
If $a$ and $b$ are integers, then there exist integers $u$ and $v$ such that $\GCD{a}{b} = ua + vb$
\end{thm}

\begin{proof}
We start with the case $b \geq 0$, proceeding by strong induction.
\begin{itemize}
\item \textbf{Base Case} ($b = 0$): Note that $\GCD{a}{0} = a = a \cdot 1 + 0 \cdot 0$ as needed. That is, the result holds with $u = 1$ and $v = 0$.
\item \textbf{Base Case} ($b = 1$): Note that $\GCD{a}{1} = 1 = a \cdot 0 + 1 \cdot 1$ as needed. That is, the result holds with $u = 0$ and $v = 1$.
\item \textbf{Inductive Step}: Suppose the result holds for all integers $b'$ with $0 \leq b' < b$, where $b > 1$. That is, for all such $b'$ and all integers $a$ there exist integers $u$ and $v$ such that $\GCD{a}{b'} = au + b'v$. Now consider $b$. By the division algorithm we have integers $q$ and $r$ such that $a = qb + r$ and $0 \leq r < b$. We have two possibilities to consider.
\begin{itemize}
\item If $r = 0$, then in fact $b|a$, since $a = qb$. So $\GCD{a}{b} = b = a \cdot 0 + q \cdot b$. That is, the result holds with $u = 0$ and $v = 1$.
\item If $r > 0$, then by the induction hypothesis there exist integers $u'$ and $v'$ such that $\GCD{b}{r} = bu' + rv'$. By the euclidean algorithm, we have
\begin{eqnarray*}
\GCD{a}{b} & = & \GCD{b}{r} \\
 & = & bu' + rv' \\
 & = & bu' + (a - qb)v' \\
 & = & av' + b(u' - qv').
\end{eqnarray*}
That is, the result holds with $u = v'$ and $v = u' - qv'$.
\end{itemize}
\end{itemize}
By Strong Induction, for all $b \geq 0$ and all integers $a$ there exist integers $u$ and $v$ such that $\GCD{a}{b} = au + bv$.

Now suppose $b < 0$, so that $-b > 0$. By the previous discussion, there exist integers $u'$ and $v'$ such that $\GCD{a}{-b} = au' + (-b)v'$. Now \[ \GCD{a}{b} = \GCD{a}{-b} = au' + (-b)v' = au' + b(-v'). \] That is, the result holds with $u = u'$ and $v = -v'$.  
\end{proof}

Similar to the Euclidean Algorithm, this proof of Bezout's Identity provides us with a strategy for actually finding the coefficients $u$ and $v$ recursively.

\begin{dfn}[Relatively Prime]
We say that integers $a$ and $b$ are \emph{relatively prime} if $\GCD{a}{b} = 1$.
\end{dfn}

\begin{thm}[Euclid's Lemma]
If $a$ and $b$ are relatively prime integers and $c$ an integer such that $a | bc$, then $a|c$.
\end{thm}

\begin{proof}
By Bezout's Identity, we have $1 = au + bv$ for some integers $u$ and $v$; so $c = auc + bvc$. Since $a|bc$, we have $bc = at$ for some integer $t$. Thus \[ c = auc + bvc = auc + atv = a(uc + tv), \] and so $a|c$ as claimed.
\end{proof}

\end{document}

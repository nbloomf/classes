\documentclass{article}
\usepackage{neb-macros}

\begin{document}

\CheapTitle{Divisibility}

\begin{dfn}
Let $R$ be a commutative, unital ring, with $a,b \in R$. We say $a$ \emph{divides} $b$, denoted $a|b$, if there is an element $c \in R$ such that $b = ac$. We say $a$ is \emph{associate to} $b$ if there is a \emph{unit} $c$ such that $b = ac$.
\end{dfn}

\begin{prop}
Let $R$ be a C.U. ring with $a,b,c \in R$.
\begin{enumerate}
\item $a|a$
\item If $a|b$ and $b|c$ then $a|c$
\item If $u$ is a unit, then $u|a$.
\item ``Is associate to'' is an equivalence relation.
\item The only associate of 0 is 0.
\item The associates of 1 are precisely the units.
\end{enumerate}
\end{prop}

\begin{prop}
If $R$ is a domain, then $a|b$ and $b|a$ if and only if $a$ and $b$ are associates.
\end{prop}

In a domain, every element is divisible by (1) units and (2) its associates. These are called \emph{trivial divisors}. In general, a ring element will have more divisors. Some ring elements, however, have \emph{only} the trivial divisors. These are special.

\begin{dfn}[Irreducible]
Let $R$ be a domain and $x \in R$ a nonzero nonunit. We say that $x$ is \emph{irreducible} in $R$ if, whenever $a,b \in R$ such that $x = ab$, either $a$ or $b$ is a unit.
\end{dfn}

\begin{center}
\framebox{Given a domain $R$, what are the irreducible elements of $R$?}
\end{center}

\subsubsection*{Examples}

\begin{itemize}
\item In $\ZZ$ the irreducible elements are precisely the prime integers.
\item If $R$ is a field, then $R$ has no irreducible elements. (There are no nonzero nonunits!)
\end{itemize}


\subsection*{Brief Aside: Norms}

For some rings (not all!) we can make progress on the problem of finding irreducibles by mapping the multiplicative structure of $R$ to the $\NN$ - doing this we can take advantage of what we know about natural numbers and, sometimes, recover the benefits of induction.

\begin{dfn}
Let $R$ be a domain. A mapping $N : R \rightarrow \NN$ is called a \emph{multiplicative norm} if $N(\alpha) = 0$ if
\begin{itemize}
\item $N(\alpha) = 0$ iff $\alpha = 0$ and
\item $N(\alpha\beta) = N(\alpha)N(\beta)$ for all $\alpha, \beta \in R$, and
\item If $N(u) = 1$, then $u$ is a unit in $R$.
\end{itemize}
\end{dfn}

\subsubsection*{Examples}

\begin{itemize}
\item $N : \ZZ \rightarrow \NN$ given by $N(a) = |a|$ is a multiplicative norm.
\item $N : \ZZ[i] \rightarrow \NN$ given by $N(a+bi) = a^2 + b^2$ is a multiplicative norm.
\item More generally, $N : \mathcal{O}(\sqrt{D}) \rightarrow \NN$ given by $N(a+b\sqrt{D}) = |a^2 + Db^2|$ if $D \equiv 2,3 \mod 4$ and $N(a+b\frac{1+\sqrt{D}}{2}) = |a^2 + ab + b^2\frac{1-D}{4}|$ if $D \equiv 1 \mod 4$ is a multiplicative norm.
\end{itemize}

Multiplicative norms allow us to detect irreducible elements.

\begin{prop}
Let $R$ be a domain and $N : R \rightarrow \NN$ a multiplicative norm. If $\alpha \in R$ such that $N(\alpha)$ is prime in $\NN$, then $\alpha$ is irreducible in $R$.
\end{prop}

For example, consider $\ZZ[i]$. Applying this result here, we see that $a \pm bi$ is irreducible if $a^2 + b^2$ is prime. In particular $1 \pm i$, $1 \pm 2i$, $2 \pm 3i$, and many other Gaussian integers are irreducible (since $1^2 + 1^2 = 2$, $1^2 + 2^2 = 5$, and $2^2 + 3^2 = 13$ are prime). This leads to a natural question about the natural numbers: for which primes $p$ does the equation $a^2 + b^2 = p$ have a solution?

As this example shows, a good multiplicative norm can turn questions in $R$ into number theory problems. This turns out to be a useful technique more generally: given a problem about some object, look for a way to map the relevant structure of that object to some other object which either is well-understood or with which we can compute things. A good strategy for solving algebraic problems is to try to reduce to number theory or to linear algebra.

\subsection*{Primes}

\begin{dfn}
Let $R$ be a domain. We say that $p \in R$ is \emph{prime} if whenever $p|ab$, either $p|a$ or $p|b$.
\end{dfn}

\begin{prop}
If $R$ is a domain, then every prime element is also irreducible.
\end{prop}

\begin{proof}
Suppose $p \in R$ is prime, and factor $p$ as $p = ab$. In particular, $p|ab$, and since $p$ is prime, WLOG we have $p|a$. Say $a = pt$. Now $p = ab = ptb$, and by cancellation, $tb = 1$. In particular $b$ is a unit. Thus $p$ is irreducible.
\end{proof}

\end{document}

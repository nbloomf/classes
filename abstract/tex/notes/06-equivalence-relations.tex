\documentclass{article}
\usepackage{neb-macros}

\begin{document}

\CheapTitle{Equivalence Relations}

Recall:

\begin{dfn}[Equivalence Relation]
Let $A$ be a set and let $\sigma \subseteq A \times A$ be a relation on $A$. We say that $\sigma$ is an \emph{equivalence relation} if the following hold.
\begin{enumerate}
\item $a \sigma a$ for all $a \in A$. (Reflexivity)
\item If $a \sigma b$ then $b \sigma a$ for all $a,b \in A$. (Symmetry)
\item If $a \sigma b$ and $b \sigma c$, then $a \sigma c$ for all $a,b,c \in A$. (Transitivity)
\end{enumerate}
If $\sigma$ is an equivalence on $A$ and $a \in A$, then the set \[ [a]_\sigma = \{ b \in A \mid a \sigma b \} \] is called the \emph{equivalence class} of $a$.
\end{dfn}

\begin{dfn}[Partition]
Let $A$ be a set, and let $P \subseteq 2^A$ be a collection of subsets of $A$. We say that $P$ is a \emph{partition} of $A$ if the following hold.
\begin{enumerate}
\item $\bigcup P = A$ ($P$ is collectively exhaustive)
\item If $C_1, C_2 \in P$ are distinct, then $C_1 \cap C_2 = \varnothing$.
\end{enumerate}
\end{dfn}

Equivalence relations and partitions are related in a fundamental way.

\begin{thm}
Let $A$ be a set.
\begin{enumerate}
\item If $\sigma$ is an equivalence relation, then the set \[ A/\sigma = \{ [a] \mid a \in A \} \] is a partition of $A$.
\item If $P$ is a partition of $A$, then the relation \[ \sigma = \{ (x,y) \mid y \in A, x \in [y] \} \] is an equivalence on $A$.
\end{enumerate}
\end{thm}

That is, given an equivalence we can build a partition, and given a partition we can build an equivalence relation.

\begin{dfn} \mbox{}
\begin{itemize}
\item Let $A$ be a set and $\sigma$ an equivalence on $A$. The mapping $\pi_\sigma : A \rightarrow A/\sigma$ given by $\varphi(a) = [a]$ is called the \emph{natural projection} of $A$ onto $A/\sigma$.
\item Let $\varphi : A \rightarrow B$ be a function. The relation \[ \KER{\varphi} = \{ (x,y) \mid \varphi(x) = \varphi(y) \} \] is an equivalence on $A$, called the \emph{kernel} of $\varphi$.
\end{itemize}
\end{dfn}

\begin{thm}[First Isomorphism Theorem for Sets]
Let $f : A \rightarrow B$ be a function. There is a unique function $\overline{f}$ such that $f = \overline{f} \circ \pi_{\KER{f}}$.
\end{thm}

\end{document}

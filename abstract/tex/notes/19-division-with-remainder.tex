\documentclass{article}
\usepackage{neb-macros}
\usepackage{framed}

\begin{document}

\CheapTitle{Division with Remainder}

In $\ZZ$, we had the extremely important Division Algorithm. This theorem states that if $a$ and $b$ are integers with $b \neq 0$, then there exists a ``quotient'' $q$ and a ``remainder'' $r$ such that $a = qb + r$, and, moreover, the remainder is not too large -- $0 \leq r < |b|$. This is the result from which most of the interesting results and algorithms in $\ZZ$ spring.

We'd like to generalize this property to integral domains. Notice that one problem is the appearance of absolute value in the bound on $r$: in general, rings do not have anything like absolute value, or a way to compare the ``sizes'' of two elements. However we did describe such a gadget for some rings: multiplicative norms. Recall that $N : R \rightarrow \NN$ is a multiplicative norm if (1) $N(x) = 0$ iff $x = 0$, (2) $N(xy) = N(x)N(y)$, and (3) if $N(x) = 1$ then $x$ is a unit. These properties do generalize the absolute value.

\begin{dfn}[Euclidean Norm]
Let $R$ be a domain.
\begin{itemize}
\item We say that a multiplicative norm $N : R \rightarrow \NN$ is a \emph{Euclidean norm} if for all $a,b \in R$ with $b \neq 0$, there exist $q,r \in R$ such that $a = qb+r$ and $0 \leq N(r) < N(b)$.
\item If there is a Euclidean norm on $R$, we say that $R$ is a \emph{Euclidean Domain}.
\end{itemize}
\end{dfn}

Of course $\ZZ$ is a Euclidean Domain with norm $N(a) = |a|$. The existence of a Euclidean norm on $R$ is very powerful. For instance, many of the nice properties of $\ZZ$ which we derived from the Division Algorithm have analogues in any Euclidean Domain. More generally, the norm allows us to recover some of the benefits of mathematical induction.

\begin{prop}
Every Euclidean Domain is also a GCD Domain.
\end{prop}

\begin{proof}
Let $R$ be a Euclidean domain with norm $N$. We want to show that for all $a \in R$, for all $b \in R$, the set $\MCD{a}{b}$ is not empty. We proceed by strong induction on $N(a)$.

\textbf{Base case.} Suppose $N(a) = 0$. Then $a = 0$, and so $b \in \MCD{a}{b}$ for all $b$.

\textbf{Inductive Step.} Let $a \in R$ and suppose that the result holds for all $a'$ with $1 \leq N(a') < N(a)$. In particular, note that $a \neq 0$. Now let $b \in R$. By the division algorithm we may decompose $b$ as $b = qa + r$, where $0 \leq N(r) < N(a)$. If $r = 0$ then $a|b$ and we have $a \in \MCD{a}{b}$. If $r \neq 0$, then by the inductive hypothesis $\varnothing \neq \MCD{r}{a} = \MCD{b-qa}{a} = \MCD{b}{a}$ as needed.
\end{proof}

\begin{prop}
Every Euclidean domain is a Unique Factorization domain.
\end{prop}

The proof for $\ZZ$ generalizes.

\begin{prop}
Every field is a Euclidean domain.
\end{prop}

\begin{proof}
Define a mapping $N : F \rightarrow \NN$ by $N(x) = 0$ if $x = 0$ and 1 if $x \neq 0$. We can see that $N$ is a Euclidean norm.
\end{proof}

\subsection*{Example: The Gaussian Integers}

\begin{prop}
$\ZZ[i]$ is a Euclidean domain under the norm $N(a+bi) = a^2 + b^2$.
\end{prop}

\begin{proof}
Let $\alpha = a_1 + a_2 i$ and $\beta = b_1 + b_2 i$ be Gaussian integers, with $\beta \neq 0$. Thinking of $\alpha$ and $\beta$ as elements of $\QQ(i)$, we have \[ \frac{\alpha}{\beta} = t_1 + t_2 i = \frac{a_1b_1 + a_2b_2}{b_1^2 + b_2^2} + \frac{a_2b_1 - a_1b_2}{b_1^2 + b_2^2} i. \] Choose integers $q_1$ and $q_2$ such that $|q_1 - t_1| \leq \frac{1}{2}$ and $|q_2 - t_2| \leq \frac{1}{2}$. (Note that this is always possible.) Let $\gamma = q_1 + q_2 i$, and let $\delta = \alpha - \gamma\beta$. Note that by construction, $\gamma$ and $\delta$ are in $\ZZ[i]$.

We now have
\begin{eqnarray*}
N(\delta) & = & N(\alpha - \gamma\beta) = N\left((\frac{\alpha}{\beta} - \gamma)\beta\right) = N(\frac{\alpha}{\beta} - \gamma)N(\beta) \\
 & = & ((q_1-t_1)^2 + (q_2-t_2)^2)N(\beta) \leq \frac{1}{2}N(\beta) < N(\beta),
\end{eqnarray*}
as needed.
\end{proof}

\begin{cor}
$\ZZ[i]$ is a GCD domain and a UFD.
\end{cor}

Here is a worked example of the division algorithm in the Gaussian integers. Let $\alpha = 10+7i$ and $\beta = 3+2i$. Now \[ \frac{\alpha}{\beta} = \frac{44}{13} + \frac{1}{13}i = (3 + \frac{5}{13}) + (0 + \frac{1}{13})i. \] Let $t_1 = 3$ and $t_2 = 0$, so that $\gamma = 3$. Now $\delta = \alpha - \gamma\beta = 1+i$. We then have $10+7i = 3(3+2i) + (1+i)$ and $N(1+i) < N(3+2i)$.

\subsection*{Exercises}

\begin{enumerate}
\item ($k$-stage Euclidean)

\item (Factorization in $\ZZ[i]$)
\end{enumerate}

\end{document}

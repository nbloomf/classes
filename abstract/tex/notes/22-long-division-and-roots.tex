\documentclass{article}
\usepackage{neb-macros}
\usepackage{tikz}
  \usetikzlibrary{patterns}

\begin{document}

\CheapTitle{Long Division and Roots}

\begin{prop}
Let $R$ be a commutative unital ring, and let $a(x), b(x) \in R[x]$ be polynomials such that $b(x) \neq 0$ and the leading coefficient of $b$ is a unit in $R$. Then there exist polynomials $q(x), r(x) \in R[x]$ such that $a(x) = q(x)b(x) + r(x)$ and either $r(x) = 0$ or $\deg{r} < \deg{b}$.
\end{prop}

\begin{proof}
If $a(x) = 0$, set $q(x) = r(x) = 0$. Suppose now that $a(x) \neq 0$; we proceed by strong induction on $\deg{a}$.
\begin{itemize}
\item \textbf{Base case.} If $\deg{a} = 0$, then $a(x) = a_0$ is a constant. If $\deg{b} = 0$, then $b(x) = b_0$ is also a constant, and in this case $b_0$ is the leading coefficient of $b$, hence a unit. Let $q(x) = a_0b_0^{-1}$ and $r(x) = 0$. If $\deg{b} > 0$, let $q(x) = 0$ and $r(x) = a_0$. Then $a(x) = q(x)b(x) + r(x)$ and we have $\deg{b} \geq 1 > 0 = \deg{r}$.
\item \textbf{Inductive Step.} Suppose the result holds for all polynomials $\overline{a}(x)$ of degree strictly less than $n$, where $n > 0$, and suppose that $a(x)$ has degree $n$. If $\deg{a} < \deg{b}$, let $q(x) = 0$ and $r(x) = a(x)$. Now suppose instead that $\deg{a} \geq \deg{b}$. Let $m = \deg{b}$ and let $a_n$ be the leading coefficient of $a(x)$ and $b_m$ the leading coefficient of $b(x)$ (which is a unit). Define $\overline{a}(x) = a(x) - a_nb_m^{-1}x^{n-m}b(x)$. Note that $\deg{\overline{a}} < \deg{a}$. By the inductive hypothesis, we have $\overline{q}(x), r(x) \in R[x]$ such that $\overline{a}(x) = \overline{q}(x)b(x) + r(x)$ and either $r(x) = 0$ or $\deg{r} < \deg{b}$. Define $q(x) = \overline{q}(x) + a_nb_m^{-1}x^{n-m}$. Now 
\begin{eqnarray*}
a(x) - q(x)b(x) & = & a(x) - \overline{q}(x)b(x) - a_nb_m^{-1}x^{n-m}b(x) \\
 & = & \overline{a}(x) - \overline{q}(x)b(x) \\
 & = & r(x)
\end{eqnarray*}
as needed.
\end{itemize}
By induction, the result holds for all $n$.
\end{proof}

\begin{cor}
Suppose $F$ is a field.
\begin{enumerate}
\item $F[x]$ is a Euclidean domain with norm $N(a) = 2^{\deg{a}}$. In particular, $F[x]$ is also a UFD and a GCD domain.
\item $p(x) \in F[x]$ is irreducible iff $p(x)$ cannot be factored as a product of nonconstants.
\item If $p(x)$ has degree 1, then $p(x)$ is irreducible in $R[x]$.
\end{enumerate}
\end{cor}

\subsection*{The Evaluation Map}

So far, we've been thinking of polynomials as objects in their own right. But we can also treat them like functions in the usual sense by ``plugging in'' ring elements for the variable. Given a polynomial $p(x) = \sum_{i=0}^n a_nx^n$ in $R[x]$, $R$ a commutative unital ring, we define the \emph{evaluation map} $\varepsilon_p : R \rightarrow R$ by $\varepsilon_p(r) = \sum_{i=0}^n a_i r^i$.

\begin{prop}[Factor Theorem]
Let $R$ be a commutative unital ring, with $p(x) \in R[x]$ and $a \in R$. Then $a$ is a root of $p(x)$ if and only if $x-a$ divides $p(x)$ in $R[x]$.
\end{prop}

\begin{proof}
Certainly if $x-a$ divides $p(x)$ then $a$ is a root of $p$. Conversely, suppose $a$ is a root of $p(x)$. Now $b(x) = x - a$ is monic, so by the polynomial long division algorithm we have $q(x), r(x) \in R[x]$ such that $p(x) = q(x)(x-a) + r(x)$ and either $r(x) = 0$ or $\deg{r} < 1$. If $r(x) \neq 0$, then $r(x) = r_0$ is a constant. Evaluating at $a$ we have $p(a) = r_0$, a contradiction. So $r(x) = 0$ and $x-a$ divides $p(x)$.
\end{proof}

\begin{prop}
Let $R$ be a domain and $p(x) \in R[x]$ a polynomial of degree 2 or 3. Then $p(x)$ cannot be written as a product of nonconstants in $R[x]$ if and only if $p(x)$ does not have a root in $R$.
\end{prop}

\begin{proof}
Note that if $p(x) = a(x)b(x)$, then $\deg{a} + \deg{b}$ is either 2 or 3. Thus $p(x)$ is a product of nonconstants iff it has a factor of degree 1. But $p(x)$ has a factor of degree 1 iff it has a root in $R$.
\end{proof}

\subsection*{Example}

\begin{itemize}
\item Show that $p(x) = x^2 + 1$ is irreducible over $\ZZ/(3)$. 
\end{itemize}

\section{Eisenstein's Irreducibility Criterion}

\begin{prop}
Let $R$ be a domain and $q(x) = \sum_{i=0}^n \in R[x]$. Suppose $p \in R$ is prime such that $p \not| a_n$, $p|a_i$ for each $0 \leq i < n$, and $p^2 \not| a_0$. Then $q(x)$ cannot be factored as a product of nonconstants. 
\end{prop}

\begin{proof}
Suppose we have \[ q(x) = b(x)c(x) = \left(\sum_{i}b_ix^i\right)\left(\sum_{j}c_jx^j\right) = \sum_{k}\left( \sum_{i+j = k}b_ic_j \right)x^k. \] Note that $q_0 = b_0c_0$. Since $p | b_0c_0$ and $p^2 \not| b_0c_0$, $p$ divides exactly one of $b_0$ and $c_0$; suppose WLOG that $p|b_0$, so $p \not| c_0$. Letting $n = \deg{q}$, $h = \deg{b}$, and $k = \deg{c}$, we have $q_n = b_hc_k$, and since $p \not| q_n$, $p \not| b_h$. Let $i$ be minimal such that $p\not|b_i$. (Note that $0 < i \leq \deg{b} \leq n$.)

We now have \[ q_i = b_ic_0 + b_{i-1}c_1 + \cdots + b_{i-t}c_t \] for some $t$. If $i < n$, then $p|q_i$, and by construction, $p|b_j$ for $j < i$. Thus $p|b_ic_0$, and since $p \not| c_0$ we have $p|b_i$ -- a contradiction. So $i = n$ and thus $\deg{b} = n = \deg{q}$. But $\deg{q} = \deg{b} + \deg{c}$, so that $\deg{c} = 0$; hence $c$ is a constant.
\end{proof}

\end{document}

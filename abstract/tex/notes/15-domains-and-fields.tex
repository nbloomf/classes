\documentclass{article}
\usepackage{neb-macros}
\usepackage{tikz}
  \usetikzlibrary{patterns}

\begin{document}

\CheapTitle{Domains and Fields}

The integers have the following very nice ``zero product property'':

\begin{center}
\framebox{If $a$ and $b$ are integers and $ab = 0$, then either $a = 0$ or $b = 0$.}
\end{center}

Recall that $\ZZ/(n)$ does not necessarily have this property. For instance, in $\ZZ/(6)$ we have $2 \neq 0$ and $3 \neq 0$, but $2 \cdot 3 = 0$. In this case we say that $2$ and $3$ are zero divisors in $\ZZ/(6)$.

\begin{dfn}[Zero Divisor]
Let $R$ be a ring.
\begin{itemize}
\item We say that a nonzero element $r \in R$ is a \emph{zero divisor} if there is a nonzero element $s \in R$ such that $rs = 0$.
\item We say that $R$ is an \emph{integral domain}, or simply \emph{domain}, if $R$ is commutative and does not contain any zero divisors.
\end{itemize}
\end{dfn}

\begin{prop}[Cancellation]
Let $R$ be a domain with $r,s,t \in R$. If $rs = rt$, then $s = t$.
\end{prop}

\subsection*{Units and Fields}

\begin{dfn}[Unit]
Let $R$ be a unital ring.
\begin{itemize}
\item We say that $u \in R$ is a \emph{unit} if there is an element $v \in R$ such that $uv = vu = 1_R$.
\item We say that $R$ is a \emph{field} if $R$ is commutative and every nonzero element of $R$ is a unit.
\end{itemize}
\end{dfn}

\begin{prop}
Every field is also an integral domain.
\end{prop}

\begin{prop}
$\ZZ/(n)$ is a field if and only if $n$ is prime.
\end{prop}


\subsection*{Exercises}

\begin{enumerate}
\item Ponder: Is the zero ring a domain?

\item Show that if $R$ and $S$ are nontrivial rings, then $R \oplus S$ is \emph{not} a domain.

\item Show that every subring of a domain is a domain.

\item Show that every subring of a field is a domain.

\item \textbf{Nilpotence.} We say that an element $r$ in a ring $R$ is \emph{nilpotent} if $r^n = 0$ for some natural number $n$.

\item \textbf{Skew fields.}
\end{enumerate}

\end{document}

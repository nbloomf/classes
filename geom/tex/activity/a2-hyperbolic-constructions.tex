\documentclass{article}
\usepackage{neb-titles}
\usepackage{neb-macros}

\pagestyle{empty}

\begin{document}

\HomeworkTitle[class={College Geometry}, number={2}, name={Poincar\'{e} Half-Plane}]

In class we began to explore a strange new model of geometry called the Poincar\'{e} Half-Plane (pronounced ``pwan kar ay''). Even though the half-plane is non-euclidean, and behaves very strangely compared to our geometric intuition, the primitive notions of ``line'', ``between'', and ``congruent'' can be understood in terms of their Euclidean counterparts, allowing us to visualize the Half-Plane on top of the Euclidean plane.

\textbf{In this activity, you will build SketchPad (or Geogebra) tools to represent the primitive constructions of lines, segments, rays, and circles in this model.} Open a fresh Sketchpad or Geogebra file and save it as "(Your Name)'s Poincare Tools". After completing the following constructions, \textbf{send me a copy to grade}.

\begin{enumerate}
\item Remember: All constructions in the Half-Plane are carried out on a fixed side of a fixed (Euclidean) line, $\mathcal{L}$. Start by constructing two points, called $P$ and $Q$, and the line $\mathcal{L}$ through them. We will call this line the ``ideal axis''.

\item \textbf{Type II Lines.} Lines in the Half-Plane come in two flavors, called Type I and Type II (these are not very creative names). The Type I lines are Euclidean half-lines which are perpendicular to the ideal axis; ``almost none'' of the lines are Type I, so we can safely ignore them for now.

Construct two points, $A$ and $B$, on the same side of the ideal axis. Construct the perpendicular bisector of $\Segment{A}{B}$, and construct the point $O$ where this line cuts the ideal axis; we will call this $O$ the \emph{ideal center} of the line generated by $A$ and $B$. Now construct the circle centered at $O$ and passing through $A$, and construct the two points $H$ and $K$ where this circle cuts the ideal axis. Finally, select the points $H$, $A$, $K$ (in this order) and select ``arc through 3 points'' from the Construct menu in Sketchpad (Geogebra has a similar tool). \textbf{This arc $\alpha$ is the Type II line generated by $A$ and $B$.}

To avoid having to carry out all these steps whenever we want to make a Type II line, we can make a tool. Select $P$, $Q$, $A$, $B$, and $\alpha$ (in this order!) and choose ``Create New Tool''. Call this tool ``Type II line from two points''. We will also create a tool to construct the point $O$ by selecting $P$, $Q$, $A$, $B$, and $O$ (in this order!); call this tool ``Ideal center of two points''.

After creating these two tools, you can delete $A$ and $B$. Test out your new tools to make sure they work as expected.



\item \textbf{Type II Segments.} Segments in the Half-Plane also come in two flavors, depending on whether their endpoints generate a Type I or a Type II line. We will only consider the Type II segments.

Construct two points, $A$ and $B$, on the same side of the ideal axis. Then construct the Type II line $\ell$ generated by $A$ and $B$, the ideal center $O$ of $A$ and $B$, and the Euclidean midpoint $M$ of $\Segment{A}{B}$. Now construct the Euclidean ray $\Ray{O}{M}$, and construct the point $T$ where this ray cuts the Type II line $\ell$. Finally, select the points $A$, $T$, and $B$ (in this order) and construct the arc through these 3 points. \textbf{This arc $\alpha$ is the Type II segment generated by $A$ and $B$.}

Make a tool to construct Type II segments by selecting $P$, $Q$, $A$, $B$, and $\alpha$ in this order. Call your tool ``Type II segment from endpoints''.

After making the segment tool, delete $A$ and $B$. Test the new tool to make sure it works as expected.



\item \textbf{Type II Rays.} Constructing rays will require us to trick Sketchpad (or Geogebra) a little bit.

Construct two points $A$ and $B$ on the same side of the ideal axis. Construct the ideal center $O$ of $A$ and $B$ and the Euclidean midpoint $M$ of $A$ and $B$, and construct the Euclidean line $\Line{O}{M}$. Now construct the (unique!) Euclidean line passing through $B$ and parallel to $\Line{O}{M}$. Construct the point $T$ where this line cuts the ideal axis. Now construct the Euclidean circle $c$ centered at $O$ and passing through $A$, and the Euclidean ray $\Ray{O}{T}$. Construct the point $U$ where these two meet. (Make sure you intersect with the ray $\Ray{O}{T}$, not the line $\Line{O}{T}$! This is the trick.) Finally, select the points $A$, $B$, and $U$ (in this order) and construct the arc through these 3 points. \textbf{This arc $\alpha$ is the Type II ray from $A$ toward $B$.}

Make a tool by selecting $P$, $Q$, $A$, $B$, and $\alpha$ (in this order), and call it ``Type II ray from two points''.

After making the ray tool, delete $A$ and $B$. Test the new tool to make sure it works as expected.



\item \textbf{Type II Circles.} Like lines, the construction of the circle centered at $O$ and passing through $A$ depends on whether or not the Euclidean line $\Line{O}{A}$ is perpendicular to the ideal axis; we distinguish these cases as Type I and Type II circles. Almost all pairs of points are in the Type II case, so we can safely ignore Type I.

Construct two points, $O$ and $A$, on the same side of the ideal axis. $O$ will be the center of the Type II circle, and $A$ will be a point on the circle. Construct the ideal center $X$ of $A$ and $O$, and construct the line $\ell$ which is perpendicular to the Euclidean line $\Line{X}{A}$ at $A$. Now construct the Euclidean line $m$ which is perpendicular to the ideal axis and passes through $O$. Let $Y$ be the point where $\ell$ and $m$ meet. Finally, construct the Euclidean circle centered at $Y$ and passing through $A$. \textbf{This circle $c$ is the Type II circle centered at $O$ and passing through $A$.}

Make a new tool by selecting $P$, $Q$, $O$, $A$, and $c$ (in this order) and call it ``Type II circle from center and point''.

After making the circle tool, delete $O$ and $A$. Test the new tool to make sure it works as expected.
\end{enumerate}


Using these four tools, all of the neutral plane geometry constructions we've seen can be carried out \textbf{without modification}: copying segments, copying angles, and so on. This is the power of a well-chosen abstraction. We will be doing this in the next homework assignment, so be sure to save your tools!

\end{document}

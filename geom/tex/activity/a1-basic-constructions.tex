\documentclass{article}
\usepackage{neb-titles}
\usepackage{neb-macros}

\pagestyle{empty}

\begin{document}

\HomeworkTitle[class={College Geometry}, number={1}, name={Basic Constructions}]

\noindent In this activity you will carry out some basic constructions using Geometer's Sketchpad (GS). If you have never used GS before, don't worry; for what we need it to do, GS is fairly straightforward. If you get stuck, ask for help!

Each of the following problems should be constructed on a separate sketchpad file, and these files should be given descriptive titles \textbf{including your name and the homework number}. When you are finished, email the sketchpad files to me. So I should get four files from you, with names like ``Nathan Bloomfield - HW1 - construct equilateral triangle.gsp''.

Finally, \textbf{your constructions must be robust}. Every geometric construction starts with one or more \emph{free elements}; these are points and lines given in the hypotheses of the construction proof. You should be able to \textbf{move the free elements around} without destroying your construction.

\begin{enumerate}
\item \textbf{Construct an equilateral triangle.} Start by placing two free points $X$ and $Y$ in the plane, as well as the segment $\Segment{X}{Y}$. Following the proof we gave in class, construct two points $Z_1$ and $Z_2$, on the opposite sides of $\Line{X}{Y}$, such that $\Triangle{X}{Y}{Z_1}$ and $\Triangle{X}{Y}{Z_2}$ are equilateral.

\item \textbf{Copy a line segment.} Start by placing four free points $X$, $Y$, $O$, and $P$ in the plane, as well as the segment $\Segment{X}{Y}$ and the ray $\Ray{O}{P}$. Following the proof we gave in class, construct a point $Z$ on $\Ray{O}{P}$ such that $\Segment{O}{Z} \equiv \Segment{X}{Y}$.

\item \textbf{Copy an angle.} Start by placing five free points $A$, $O$, $B$, $P$, and $X$ in the plane, as well as the rays $\Ray{O}{A}$, $\Ray{O}{B}$, and $\Ray{P}{X}$. Following the proof we gave in class, construct two points $Y_1$ and $Y_2$, on opposite sides of $\Line{O}{X}$, such that $\Angle{X}{P}{Y_1} \equiv \Angle{A}{O}{B}$ and $\Angle{X}{P}{Y_2} \equiv \Angle{A}{O}{B}$.

\item \textbf{Construct a midpoint.} Start by placing two free points $X$ and $Y$ in the plane, as well as the segment $\Segment{X}{Y}$. Construct a point $Z \in \Segment{X}{Y}$ such that $\Segment{X}{Z} \equiv \Segment{Z}{Y}$. Can you prove that the point you constructed has this property?
\end{enumerate}

\end{document}

\documentclass{article}
\usepackage{neb-macros}
\usepackage{tikz}
  \usetikzlibrary{calc,intersections}

\begin{document}

\CheapTitle{Incircles and Excircles}

\begin{prop}
Let $A$, $O$, and $B$ be distinct points. A point $P$ in $\IntAngle{A}{O}{B}$ is on the bisector of $\Angle{A}{O}{B}$ if and only if $\Segment{P}{X} \equiv \Segment{P}{Y}$, where $X$ is the foot of $P$ on $\Line{O}{A}$ and $Y$ is the foot of $P$ on $\Line{O}{B}$.
\end{prop}

\begin{proof}
Suppose $P$ has this property. Now $\Triangle{O}{P}{X}$ and $\Triangle{O}{P}{Y}$ are right, with $\Segment{P}{X} \equiv \Segment{P}{Y}$ and $\Segment{O}{P} \equiv \Segment{O}{P}$. By the HL Theorem, $\Triangle{O}{P}{X} \equiv \Triangle{O}{P}{Y}$, and thus $\Angle{X}{O}{P} \equiv \Angle{Y}{O}{P}$. So $P$ is on the bisector of $\Angle{A}{O}{B}$.

Conversely, suppose $P$ is on the bisector of $\Angle{A}{O}{P}$, and let $X$ be the foot of $P$ on $\Line{O}{A}$ and $Y$ the foot of $P$ on $\Line{O}{B}$. Now $\Triangle{X}{O}{P} \equiv \Triangle{Y}{O}{P}$ by AAS, so that $\Segment{P}{X} \equiv \Segment{P}{Y}$. 
\end{proof}

\begin{construct}[Incircle Theorem]
Let $A$, $B$, and $C$ be distinct points. Then we have the following.
\begin{enumerate}
\item The bisectors of the interior angles of $\Triangle{A}{B}{C}$ are concurrent at a point $O$, called the \emph{incenter} of the triangle.

\item The feet of $O$ on the sides of $\Triangle{A}{B}{C}$ lie on a circle, called the \emph{incircle} of $\Triangle{A}{B}{C}$, which is centered at $O$ and tangent to the sides of $\Triangle{A}{B}{C}$.
\end{enumerate}
\end{construct}

\begin{proof}
Let $\Ray{A}{A'}$ be the bisector of $\Angle{B}{A}{C}$. By the Crossbar Theorem this ray cuts $\Segment{B}{C}$ at a point $A''$. Let $\Ray{B}{B'}$ be the bisector of $\Angle{A}{B}{C}$; again by the Crossbar Theorem this ray cuts $\Segment{A}{A''}$ at a point $O$. Let $X$, $Y$, and $Z$ be the feet of $O$ on $\Line{A}{C}$, $\Line{A}{B}$, and $\Line{B}{C}$, respectively. Since $O$ is on the bisectors of $\Angle{B}{A}{C}$ and $\Angle{A}{B}{C}$, we have $\Segment{O}{X} \equiv \Segment{O}{Y}$ and $\Segment{O}{Y} \equiv \Segment{O}{Z}$; thus $\Segment{O}{X} \equiv \Segment{O}{Z}$, and so $O$ is also on the bisector of $\Angle{B}{C}{A}$. Thus the bisectors of the interior angles of $\Triangle{A}{B}{C}$ are concurrent at $O$.

Now $X$, $Y$, and $Z$ are the feet of $O$ on the sides of $\Triangle{A}{B}{C}$, and we've seen that $\Segment{O}{X} \equiv \Segment{O}{Y} \equiv \Segment{O}{Z}$. Thus the circle $\Circle{O}{X}$ contains $X$, $Y$, and $Z$, and moreover is tangent to the sides of $\Triangle{A}{B}{C}$ at $X$, $Y$, and $Z$.
\end{proof}

\begin{construct}[Excircle Theorem]
Let $A$, $B$, and $C$ be distinct points forming $\Triangle{A}{B}{C}$. Then we have the following.
\begin{enumerate}
\item The bisector of the interior angle at $A$ and the exterior angles at $B$ and $C$ are concurrent at a point $O$, called the \emph{excenter} of $\Triangle{A}{B}{C}$ at $A$.

\item The feet of $O$ on the (extended) sides of $\Triangle{A}{B}{C}$ lie on a circle, called the \emph{excircle} of $\Triangle{A}{B}{C}$ at $A$, which is centered at $O$ and tangent to the sides of $\Triangle{A}{B}{C}$.
\end{enumerate}
\end{construct}

\begin{proof}
Essentially the same as the proof of the Incircle Theorem.
\end{proof}

To every triangle we can associate four special circles: the incircle, and one excircle for each vertex. These circles are tangent to all three (extended) sides of the circle.

\begin{prop}
Any circle which is tangent to all three (extended) sides of a triangle is either the incircle or one of the excircles.
\end{prop}

\end{document}

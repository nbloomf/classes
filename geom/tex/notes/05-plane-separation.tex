\documentclass{article}
\usepackage{neb-macros}

\begin{document}

\CheapTitle{The Plane Separation Property}

\begin{dfn}[Convexity]
Let $\mathcal{P}$ be an incidence geometry with a betweenness relation $\Between{\cdot}{\cdot}{\cdot}$. A non empty set $S$ of points in $\mathcal{P}$ is called \emph{convex} if whenever $x,y \in S$ are distinct points, $\Segment{x}{y} \subseteq S$.
\end{dfn}

\begin{dfn}[Plane Separation Property]
Let $\mathcal{P}$ be an incidence geometry with a betweenness relation $\Between{\cdot}{\cdot}{\cdot}$. We say that this geometry has the \emph{Plane Separation Property} if every line $\ell$ partitions the set of points not on $\ell$ into two nonempty, disjoint, convex sets, $H_1$ and $H_2$, with the property that if $x \in H_1$ and $y \in H_2$ then $\Segment{x}{y} \cap \ell = \{p\}$ for some point $p$. The sets $H_1$ and $H_2$ are called \emph{half-planes}.
\end{dfn}

\subsection*{Examples}

To show that a particular incidence geometry has the plane separation property, given any line we must specify the half-planes $H_1$ and $H_2$ and \emph{show} that they are nonempty, disjoint, convex sets, which have the intersection property.

\begin{itemize}
\item[$\Reals^2$] Given a line $\ell = \Line{A}{B}$, we define two half-planes as follows: \[ H_1 = \left\{ X = (x_1, x_2) \mid \DET \begin{bmatrix} a_1 & a_2 & 1 \\ b_1 & b_2 & 1 \\ x_1 & x_2 & 1 \end{bmatrix} > 0 \right\} \] and \[ H_2 = \left\{ X = (x_1, x_2) \mid \DET \begin{bmatrix} a_1 & a_2 & 1 \\ b_1 & b_2 & 1 \\ x_1 & x_2 & 1 \end{bmatrix} < 0 \right\}. \] Certainly both $H_1$ and $H_2$ are not empty, and they are disjoint by construction.

To see that $H_1$ is convex, suppose BWOC that we have points $X,Y \in H_1$ and a point $Z = (z_1, z_2)$ such that $\Between{X}{Z}{Y}$ and $Z \notin H_1$. Now \[ m = \DET \begin{bmatrix} a_1 & a_2 & 1 \\ b_1 & b_2 & 1 \\ z_1 & z_2 & 1 \end{bmatrix} \] is either 0 or negative. If $m = 0$, then in fact $Z \in \Line{A}{B}$. Since $X,Y \notin \Line{A}{B}$, we have that $\Segment{X}{Y}$ and $\Line{A}{B}$ meet at a single point $Z$; but we've seen this can only happen if $X \in H_1$ and $Y \in H_2$ (or vice versa). Suppose instead that $m < 0$; that is, $Z \in H_2$. Now we have that $\Segment{X}{Z}$ and $\Segment{Y}{Z}$ each intersect $\Line{A}{B}$ at unique points, say $W$ and $V$, respectively. Note that $\Between{X}{W}{Z}$ and $\Between{Y}{V}{Z}$. Since $\Between{X}{Z}{Y}$, we have that $X$, $Y$, $Z$, $W$, and $V$ are all collinear. If $W$ and $V$ are distinct points, then in fact $X, Y \in \Line{W}{V} = \Line{A}{B}$, a contradiction. If $W = V$, then we have $\Between{X}{W}{Z}$ and $\Between{Y}{W}{Z}$, so by the 4-point axiom, $\Between{W}{Z}{Y}$, a contradiction. So we must have $Z \in H_1$, and thus $H_1$ is convex. A similar argument shows that $H_2$ is convex.

Finally, we need to show that if $X \in H_1$ and $Y \in H_2$, then $\Segment{X}{Y} \cap \Line{A}{B}$ consists of a unique point. We showed precisely this previously.
\end{itemize}

\subsection*{Ordered Geometries}

\begin{dfn}[Ordered Geometry]
Let $\mathcal{P}$ be an incidence geometry with a betweenness relation $\Between{\cdot}{\cdot}{\cdot}$. We say that $\mathcal{P}$ (with this betweenness relation) is an \emph{Ordered Geometry} if it has the Trichotomy Property, the 4-Point Property, the Interpolation Property, and the Line Separation Property.
\end{dfn}

For example, both $\Reals^2$ and $\Rats^2$ are ordered geometries.

\begin{dfn}[Triangle]
Let $\mathcal{P}$ be an incidence geometry, and let $x$, $y$, and $z$ be distinct points. Then the set \[ \Triangle{x}{y}{z} = \Segment{x}{y} \cup \Segment{y}{z} \cup \Segment{z}{x} \] is called the \emph{triangle} with \emph{vertices} $x$, $y$, and $z$. The segments $\Segment{x}{y}$, $\Segment{y}{z}$, and $\Segment{z}{x}$ are called the \emph{sides} of the triangle.
\end{dfn}

\begin{thm}[Pasch's Axiom]
Let $x$, $y$, and $z$ be distinct points in an ordered geometry, and let $\ell$ be a line such that $x,y,z \notin \ell$. Finally, suppose there is a point $w \in \ell$ such that $\Between{x}{w}{y}$; that is, $\ell$ cuts the side $\Segment{x}{y}$.

Then precisely one of the following two things happens:
\begin{enumerate}
\item $\ell$ cuts $\Segment{y}{z}$ and does not cut $\Segment{z}{x}$, or
\item $\ell$ cuts $\Segment{z}{x}$ and does not cut $\Segment{y}{z}$.
\end{enumerate}
\end{thm}

\begin{proof}
Since $\mathcal{P}$ is an ordered geometry, it satisfies the Plane Separation property. In particular, the points not on $\ell$ are partitioned into two convex, nonempty half-planes, $H_1$ and $H_2$. Since $\Segment{x}{y} \cap \ell = \{w\}$ is not empty, without loss of generality we have $x \in H_1$ and $y \in H_2$. Since $z \notin \ell$, there are two possibilities: either $z \in H_1$ or $z \in H_2$. In the first case, we see that $\ell$ cuts $\Segment{y}{z}$ and does not cut $\Segment{z}{x}$, and in the second case, $\ell$ cuts $\Segment{z}{x}$ but not $\Segment{y}{z}$.
\end{proof}

In other words, Pasch's Axiom states that if a line enters a triangle then it must also exit.

\begin{lem}
Let $\ell$ be a line and $C \in \ell$ a point in an ordered geometry. Suppose $A$ and $B$ are points not on $\ell$ such that $\Between{A}{B}{C}$. Then $A$ and $B$ are on the same side of $\ell$.
\end{lem}

\begin{proof}
Suppose otherwise that $A$ and $B$ are on opposite sides of $\ell$. By the Plane Separation property, and because $A$ and $B$ are not on $\ell$, the segment $\Segment{A}{B}$ cuts $\ell$ at a unique point $D$. That is, $D \in \ell$ and $\Between{A}{D}{B}$. But note that $C, D \in \ell$, so $\Line{C}{D} = \ell$, and also $C, D \in \Line{A}{B}$, so that $\Line{C}{D} = \Line{A}{B}$. But then $\Line{A}{B} = \ell$, a contradiction. Thus $A$ and $B$ must be on the same side of $\ell$.
\end{proof}

\end{document}

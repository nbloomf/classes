\documentclass{article}
\usepackage{neb-macros}

\begin{document}

\CheapTitle{Betweenness}

\begin{dfn}[Betweenness]
Let $\mathcal{P}$ be an incidence geometry. We say that a ternary relation $\Between{\cdot}{\cdot}{\cdot}$ on the set of points of $\mathcal{P}$ is a \emph{betweenness relation} if the following properties hold.
\begin{itemize}
\item[B1.] If $\Between{x}{y}{x}$, then $x = y$, for all points $x$ and $y$.
\item[B2.] If $x$ and $y$ are distinct points and $\Between{x}{z}{y}$, then $\Between{y}{z}{x}$ and $z \in \Line{x}{y}$.
\item[B3.] If $x$, $y$, and $z$ are distinct points, then at most one of $\Between{x}{y}{z}$, $\Between{y}{z}{x}$, and $\Between{z}{x}{y}$ is true. 
\end{itemize}
\end{dfn}

\begin{dfn}[Segment, Ray]
Let $x$ and $y$ be distinct points in an incidence geometry $\mathcal{P} = (P,L)$.
\begin{itemize}
\item The set \[ \Segment{x}{y} = \{ z \in P \mid z = x\ \mathrm{or}\ z = y\ \mathrm{or}\ \Between{x}{z}{y} \} \] is called the \emph{segment} with \emph{endpoints} $x$ and $y$. If $z \in \Segment{x}{y}$ and $z \neq x$ and $z \neq y$, we say that $z$ is \emph{interior to} $\Segment{x}{y}$.
\item The set \[ \Ray{x}{y} = \{ z \in P \mid z = x\ \mathrm{or}\ z = y\ \mathrm{or}\ \Between{x}{z}{y}\ \mathrm{or}\ \Between{x}{y}{z} \} \] is called the \emph{ray} with \emph{vertex} $x$ \emph{toward} $y$.
\end{itemize}
\end{dfn}

\begin{prop}
If $\mathcal{P}$ is an incidence geometry and $\Between{\cdot}{\cdot}{\cdot}$ a betweenness relation on $\mathcal{P}$, then the following hold.
\begin{enumerate}
\item $\Segment{x}{y} = \Segment{y}{x}$ for all distinct points $x$ and $y$.
\item $\Segment{x}{y} \subseteq \Ray{x}{y} \subseteq \Line{x}{y}$ for all distinct points $x$ and $y$.
\item If $\ell$ is a line and $x$ and $y$ distinct points, then $\Segment{x}{y} \cap \ell$ is either $\Segment{x}{y}$, $\varnothing$, or $\{p\}$ for some point $p$.
\item $\Ray{x}{y} \cap \Ray{y}{x} = \Segment{x}{y}$ for all distinct points $x$ and $y$.
\end{enumerate}
\end{prop}



\subsection*{Examples}

\begin{itemize}
\item[$\Reals^2$] Given points $A$, $B$, and $C$ in $\Reals^2$, we say $\Between{A}{C}{B}$ if the equation $C = A + t(B-A)$ has a solution $t \in [0,1]$. This is a betweenness relation.
\begin{itemize}
\item[B1.] Suppose $\Between{A}{B}{A}$. Now $B = A + t(A - A) = A$ as needed.
\item[B2.] Suppose $A$, $B$, and $C$ are distinct points such that $\Between{A}{C}{B}$. By definition, we have $C = A + t(B-A)$ for some real number $t \in [0,1]$. Certainly $C \in \Line{A}{B}$. Moreover, note that
\begin{eqnarray*}
B + (1-t)(A-B) & = & B + A - B - t(A-B) \\
 & = & A + t(B-A) \\
 & = & C,
\end{eqnarray*}
so that $\Between{B}{C}{A}$.
\item[B3.] Suppose we have distinct points $A$, $B$, and $C$ such that $\Between{A}{B}{C}$ and $\Between{B}{C}{A}$.
Now $B = A + t(C-A)$ and $C = B + u(A-B)$ for some real numbers $u,t \in [0,1]$ by definition.
Substituting the second equation into the first, we see that $B = A + t(1-u)(B - A)$, so that $0 = (t(1-u) - 1)(B - A)$. Since $A$ and $B$ are distinct, we must have $t(1-u) = 1$. Similarly, substituting the first equation into the second, we have $u(1-t) = 1$. Then $t$ must be a root of the quadratic $t^2 - t + 1$, which has no real solutions.
\end{itemize}
\end{itemize}



\subsection*{The Trichotomy Property}

\begin{dfn}
We say that a betweenness relation $\Between{\cdot}{\cdot}{\cdot}$ on an incidence geometry $\mathcal{P}$ has the \emph{Trichotomy Property} if, whenever $x$, $y$, and $z$ are distinct, collinear points, at least one of $\Between{x}{y}{z}$, $\Between{y}{z}{x}$, and $\Between{z}{x}{y}$ is true. That is, given three collinear points, exactly one is between the other two.
\end{dfn}

\begin{prop}
Suppose $\mathcal{P}$ is an incidence geometry and $\Between{\cdot}{\cdot}{\cdot}$ a betweenness relation with the Trichotomy Property. Then the following hold.
\begin{enumerate}
\item For all distinct points $x$ and $y$, \[ \Line{x}{y} = \{ z \mid z = x\ \mathrm{or}\ z = y\ \mathrm{or}\ \Between{z}{x}{y}\ \mathrm{or}\ \Between{x}{z}{y}\ \mathrm{or}\ \Between{x}{y}{z} \}. \]
\item $\Ray{x}{y} \cap \Ray{y}{x} = \Segment{x}{y}$ for all distinct points $x$ and $y$.
\end{enumerate}
\end{prop}

\subsection*{Examples}

\begin{itemize}
\item[$\Reals^2$] The Cartesian Plane has the Trichotomy Property, as we show. Let $A$, $B$, and $C$ be distinct collinear points. Now $C \in \Line{A}{B}$, so that $C = A + t(B-A)$ for some real number $t$. If $t \in [0,1]$, then $\Between{A}{C}{B}$. If $t > 1$, then $\frac{1}{t} \in (0,1)$, and we have $B = A + \frac{1}{t}(C-A)$ so that $\Between{A}{B}{C}$. If $t < 0$, then $\frac{-t}{1-t} \in (0,1]$ and we have $A = C + \frac{-t}{1-t}(B-C)$, so that $\Between{C}{A}{B}$.
\end{itemize}



\subsection*{The 4-Point Property}

First for some shorthand: if $x$, $y$, $z$, and $w$ are distinct points, we will say $[xyzw]$ precisely when $\Between{x}{y}{z}$, $\Between{x}{y}{w}$, $\Between{x}{z}{w}$, and $\Between{y}{z}{w}$. More generally, if $x_1, \ldots, x_n$ are distinct points, then $[x_1x_2 \ldots x_n]$ means that $\Between{x_i}{x_j}{x_k}$ for all triples $(i,j,k)$ with $1 \leq i < j < k \leq n$.

\begin{dfn}[The 4-Point Property]
We say that a betweenness relation $\Between{\cdot}{\cdot}{\cdot}$ on an incidence geometry $\mathcal{P}$ has the \emph{4-Point Property} if the following hold for all distinct points $x$, $y$, $z$, and $w$.
\begin{enumerate}
\item If $\Between{x}{y}{z}$ and $\Between{x}{z}{w}$, then $\Between{x}{y}{w}$ and $\Between{y}{z}{w}$.
\item If $\Between{x}{y}{z}$ and $\Between{y}{z}{w}$, then $\Between{x}{y}{z}$ and $\Between{x}{z}{w}$.
\end{enumerate}
\end{dfn}

\begin{prop}
Suppose $\mathcal{P}$ is an incidence geometry and $\Between{\cdot}{\cdot}{\cdot}$ a betweenness relation on $\mathcal{P}$ having the 4-Point Property. If $x$, $y$, and $z$ are distinct points such that $\Between{x}{y}{z}$, then the following hold.
\begin{enumerate}
\item $\Segment{x}{y} \cup \Segment{y}{z} = \Segment{x}{z}$
\item $\Segment{x}{y} \cap \Segment{y}{z} = \{y\}$
\item $\Ray{y}{x} \cap \Ray{y}{z} = \{y\}$
\item $\Ray{x}{y} = \Ray{x}{z}$
\end{enumerate}
\end{prop}

\begin{prop}
If $\mathcal{P}$ is an incidence geometry with a betweenness relation having both the Trichotomy Property and the 4-Point Property, then the following hold.
\begin{enumerate}
\item If $\Between{x}{z}{y}$ and $\Between{x}{w}{y}$, then either $\Between{x}{z}{w}$ or $\Between{x}{w}{z}$ or $z = w$.
\item If $x$, $y$, and $z$ are distinct points such that $\Between{x}{y}{z}$, then $\Ray{y}{x} \cup \Ray{y}{z} = \Line{x}{z}$.
\end{enumerate}
\end{prop}



\subsection*{The Interpolation Property}

\begin{dfn}
We say that a betweenness relation $\Between{\cdot}{\cdot}{\cdot}$ on an incidence geometry $\mathcal{P}$ has the \emph{Interpolation Property} if for all distinct points $x$ and $y$ in $\mathcal{P}$, there exist points $z_1$, $z_2$, and $z_3$ such that $\Between{z_1}{x}{y}$, $\Between{x}{z_2}{y}$, and $\Between{x}{y}{z_3}$.
\end{dfn}

\begin{prop}
If $\mathcal{P}$ is an incidence geometry with a betweenness relation having both the Interpolation Property and the 4-Point Property, then every line in $\mathcal{P}$ has infinitely many points.
\end{prop}

\end{document}

\documentclass{article}
\usepackage{neb-macros}

\begin{document}

\CheapTitle{Betweenness}

\begin{dfn}[Betweenness]
Let $\mathcal{P}$ be an incidence geometry. We say that a ternary relation $\Between{\cdot}{\cdot}{\cdot}$ on the set of points of $\mathcal{P}$ is a \emph{betweenness relation} if the following properties hold.
\begin{itemize}
\item[B1.] If $\Between{x}{y}{x}$, then $x = y$, for all points $x$ and $y$.
\item[B2.] If $x$ and $y$ are distinct points and $\Between{x}{z}{y}$, then $\Between{y}{z}{x}$ and $z \in \Line{x}{y}$.
\item[B3.] If $x$, $y$, and $z$ are distinct points, then at most one of $\Between{x}{y}{z}$, $\Between{y}{z}{x}$, and $\Between{z}{x}{y}$ is true. 
\end{itemize}
\end{dfn}

\begin{dfn}[Segment, Ray]
Let $x$ and $y$ be distinct points in an incidence geometry $\mathcal{P} = (P,L)$.
\begin{itemize}
\item The set \[ \Segment{x}{y} = \{ z \in P \mid z = x\ \mathrm{or}\ z = y\ \mathrm{or}\ \Between{x}{z}{y} \} \] is called the \emph{segment} with \emph{endpoints} $x$ and $y$. If $z \in \Segment{x}{y}$ and $z \neq x$ and $z \neq y$, we say that $z$ is \emph{interior to} $\Segment{x}{y}$.
\item The set \[ \Ray{x}{y} = \{ z \in P \mid z = x\ \mathrm{or}\ z = y\ \mathrm{or}\ \Between{x}{z}{y}\ \mathrm{or}\ \Between{x}{y}{z} \} \] is called the \emph{ray} with \emph{vertex} $x$ \emph{toward} $y$.
\end{itemize}
\end{dfn}

\begin{prop}
If $\mathcal{P}$ is an incidence geometry and $\Between{\cdot}{\cdot}{\cdot}$ a betweenness relation on $\mathcal{P}$, then the following hold.
\begin{enumerate}
\item $\Segment{x}{y} = \Segment{y}{x}$ for all distinct points $x$ and $y$.
\item $\Segment{x}{y} \subseteq \Ray{x}{y} \subseteq \Line{x}{y}$ for all distinct points $x$ and $y$.
\item If $\ell$ is a line and $x$ and $y$ distinct points, then $\Segment{x}{y} \cap \ell$ is either $\Segment{x}{y}$, $\varnothing$, or $\{p\}$ for some point $p$.
\item $\Ray{x}{y} \cap \Ray{y}{x} = \Segment{x}{y}$ for all distinct points $x$ and $y$.
\end{enumerate}
\end{prop}



\subsection*{Examples}

\begin{itemize}
\item[$\Reals^2$]

\item[$\mathcal{A}$] 
\end{itemize}



\subsection*{The Trichotomy Property}

\begin{dfn}
We say that a betweenness relation $\Between{\cdot}{\cdot}{\cdot}$ on an incidence geometry $\mathcal{P}$ has the \emph{Trichotomy Property} if, whenever $x$, $y$, and $z$ are distinct, collinear points, exactly one of $\Between{x}{y}{z}$, $\Between{y}{z}{x}$, and $\Between{z}{x}{y}$ is true. That is, given three collinear points, exactly one is between the other two.
\end{dfn}

\begin{prop}
Suppose $\mathcal{P}$ is an incidence geometry and $\Between{\cdot}{\cdot}{\cdot}$ a betweenness relation with the Trichotomy Property. Then the following hold.
\begin{enumerate}
\item For all distinct points $x$ and $y$, \[ \Line{x}{y} = \{ z \mid z = x\ \mathrm{or}\ z = y\ \mathrm{or}\ \Between{z}{x}{y}\ \mathrm{or}\ \Between{x}{z}{y}\ \mathrm{or}\ \Between{x}{y}{z} \}. \]
\item $\Ray{x}{y} \cap \Ray{y}{x} = \Segment{x}{y}$ for all distinct points $x$ and $y$.
\end{enumerate}
\end{prop}



\subsection*{The 4-Point Property}

First for some shorthand: if $x$, $y$, $z$, and $w$ are distinct points, we will say $[xyzw]$ precisely when $\Between{x}{y}{z}$, $\Between{x}{y}{w}$, $\Between{x}{z}{w}$, and $\Between{y}{z}{w}$. More generally, if $x_1, \ldots, x_n$ are distinct points, then $[x_1x_2 \ldots x_n]$ means that $\Between{x_i}{x_j}{x_k}$ for all triples $(i,j,k)$ with $1 \leq i < j < k \leq n$.

\begin{dfn}[The 4-Point Property]
We say that a betweenness relation $\Between{\cdot}{\cdot}{\cdot}$ on an incidence geometry $\mathcal{P}$ has the \emph{4-Point Property} if the following hold for all distinct points $x$, $y$, $z$, and $w$.
\begin{enumerate}
\item If $\Between{x}{y}{z}$ and $\Between{x}{z}{w}$, then $\Between{x}{y}{w}$ and $\Between{y}{z}{w}$.
\item If $\Between{x}{y}{z}$ and $\Between{y}{z}{w}$, then $\Between{x}{y}{z}$ and $\Between{x}{z}{w}$.
\end{enumerate}
\end{dfn}

\begin{prop}
Suppose $\mathcal{P}$ is an incidence geometry and $\Between{\cdot}{\cdot}{\cdot}$ a betweenness relation on $\mathcal{P}$ having the 4-Point Property. If $x$, $y$, and $z$ are distinct points such that $\Between{x}{y}{z}$, then the following hold.
\begin{enumerate}
\item $\Segment{x}{y} \cup \Segment{y}{z} = \Segment{x}{z}$
\item $\Segment{x}{y} \cap \Segment{y}{z} = \{y\}$
\item $\Ray{y}{x} \cap \Ray{y}{z} = \{y\}$
\item $\Ray{x}{y} = \Ray{x}{z}$
\end{enumerate}
\end{prop}

\begin{prop}
If $\mathcal{P}$ is an incidence geometry with a betweenness relation having both the Trichotomy Property and the 4-Point Property, and if $x$, $y$, and $z$ are distinct points such that $\Between{x}{y}{z}$, then $\Ray{y}{x} \cup \Ray{y}{z} = \Line{x}{z}$.
\end{prop}



\subsection*{The Interpolation Property}

\begin{dfn}
We say that a betweenness relation $\Between{\cdot}{\cdot}{\cdot}$ on an incidence geometry $\mathcal{P}$ has the \emph{Interpolation Property} if for all distinct points $x$ and $y$ in $\mathcal{P}$, there exist points $z_1$, $z_2$, and $z_3$ such that $\Between{z_1}{x}{y}$, $\Between{x}{z_2}{y}$, and $\Between{x}{y}{z_3}$.
\end{dfn}

\begin{prop}
If $\mathcal{P}$ is an incidence geometry with a betweenness relation having both the Interpolation Property and the 4-Point Property, then every line in $\mathcal{P}$ has infinitely many points.
\end{prop}

\end{document}

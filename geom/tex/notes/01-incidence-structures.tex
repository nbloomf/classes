\documentclass{article}
\usepackage{neb-macros}

\begin{document}

\CheapTitle{Incidence Structures}

\begin{dfn}[Incidence Structure]
Let $P$ be a set and let $L \subseteq 2^P$ be any collection of subsets of $P$. Then $\mathcal{P} = (P,L)$ is called an \emph{incidence structure}.
\end{dfn}

If $(P,L)$ is an incidence structure, then we also make the following definitions.
\begin{itemize}
\item Elements of $P$ are called \emph{points}.
\item Elements of $L$ are called \emph{lines}.
\item If $p$ is a point and $\ell$ a line such that $p \in \ell$, we say that $p$ \emph{lies on} $\ell$ or that $\ell$ \emph{contains} $p$.
\item Given a set $S \subseteq P$ of points, we say that $S$ is \emph{collinear} if there is a line $\ell$ such that $S \subseteq \ell$.
\end{itemize}



\subsection*{Examples}

To define an incidence structure, we need to specify two things: the set of points, and which sets of points are considered to be lines. It is important to remember that ``point'' and ``line'' here are just words! Exactly what they mean depends on context. Here are some incidence structures that we will use as examples.

\begin{itemize}
\item[$2^P$] \textbf{Trivial Incidence Structures.} Let $P$ be \emph{any} set, and let $L = 2^P$ be the collection of all subsets of $P$. Certainly $(P, 2^P)$ is an incidence structure. This example is not terribly interesting, however, since every set of points is collinear. To be really useful, collinearity should say something very strong about a set of points -- this structure has too many lines.


\item[$\Reals^2$] \textbf{The Cartesian Plane.} Let $P = \Reals^2 = \{ (x,y) \mid x,y \in \Reals \}$. Now elements of $\Reals^2$ have a kind of arithmetic as follows.
\begin{eqnarray*}
(x_1, y_1) + (x_2, y_2) & = & (x_1 + x_2, y_1 + y_2) \\
t(x,y) & = & (tx, ty)\ \mathrm{if}\ t \in \Reals
\end{eqnarray*}
(This may look familiar as the vector space structure on $\Reals^2$.) Given distinct points $A,B \in \Reals^2$, we define the \emph{line generated by} $A$ and $B$ as follows. \[ \ell_{A,B} = \{ A + t(B-A) \mid t \in \Reals \}. \] Finally, let $L = \{ \ell_{A,B} \mid A,B \in \Reals^2, A \neq B \}$ be the set of all such lines. We will call the incidence structure $\Reals^2 = (P,L)$ the \emph{Cartesian plane}.

As an exercise, can you find a succinct description of the line generated by $(0,1)$ and $(1,2)$?

More generally, we can give a succinct description of any such line.

\begin{prop}
Given distinct points $A = (a_1, a_2)$ and $B = (b_1, b_2)$ in the Cartesian plane, the line $\ell_{A,B}$ is precisely the set of points $X = (x, y)$ which satisfy the equation \[ x(b_2-a_2) - y(b_1-a_1) = a_1b_2 - a_2b_1. \]
\end{prop}

\begin{proof}
If $X \in \ell_{A,B}$, then $X = A + t(B-A)$ for some real number $t$. Comparing entries, we have $x = a_1 + t(b_1 - a_1)$ and $y = a_2 + t(b_2 - a_2)$. Solving each equation for $t$, we see that \[ \frac{x-a_1}{b_1-a_1} = \frac{y-a_2}{b_2-a_2}. \] We can rearrange this equation as needed.

Conversely, suppose $(x,y)$ is a solution of the equation above; we can rearrange so that \[ \frac{x-a_1}{b_1-a_1} = \frac{y-a_2}{b_2-a_2} = t; \] it is straightforward to show that $A + t(B-A) = X$, so that $X \in \ell_{A,B}$.
\end{proof}

In fact, this leads to another useful characterization of the points on $\ell_{A,B}$.

\begin{cor}
If $A$ and $B$ are distinct points, then $(x_1,x_2) \in \ell_{A,B}$ if and only if \[ \DET \begin{bmatrix} a_1 & a_2 & 1 \\ b_1 & b_2 & 1 \\ x_1 & x_2 & 1 \end{bmatrix} = 0. \]
\end{cor}


\item[$\mathbb{D}$] \textbf{The Unit Disk.} Let $P = \{ (x,y) \mid x,y \in \Reals, x^2 + y^2 < 1 \}$; these are points in the Cartesian plane whose distance from the origin is less than 1. Given distinct points $A,B \in \mathbb{D}$, we define the \emph{line generated by} $A$ and $B$ as follows. \[ \ell_{A,B}^\mathbb{D} = \ell_{A,B} \cap \mathbb{D} \] That is, a ``line'' is an ordinary Cartesian line intersected with the unit disk. Then let $L$ be the set of all such lines. We will call the incidence structure $\mathbb{D} = (P,L)$ the \emph{Unit Disk}.


\item[$F$] \textbf{The Fano Plane.} Let $P = \{1,2,3,4,5,6,7\}$, and then let $L = \{\{1,2,3\},$ $\{2,4,6\},$ $\{1,4,7\},$ $\{1,5,6\},$ $\{2,5,7\},$ $\{3,4,5\},$ $\{3,6,7\}\}$. We call the incidence structure $\mathcal{F} = (P,L)$ the \emph{Fano plane}.


\item[$\Rats^2$] \textbf{The Rational Plane.} Similar to the Cartesian plane, let $P = \Rats^2$, and given two distinct rational points $A$ and $B$ define the line generated by $A$ and $B$ to be \[\ell_{A,B} = \{ A + t(B-A) \mid t \in \Rats \}. \] Let $L$ be the set of all such lines. Then the incidence structure $\Rats^2 = (P,L)$ is called the \emph{rational plane}. Note that lines in $\Rats^2$ look much like lines in $\Reals^2$ except that they are filled with ``holes''; any point on a line in $\Reals^2$ which has an irrational coordinate is not on the corresponding line in $\Rats^2$.


\item[$\Reals^3$] \textbf{Three-Space.} Also similar to the Cartesian plane, let $P = \Reals^3$, and given two distinct triples $A$ and $B$ define the line generated by $A$ and $B$ to be \[\ell_{A,B} = \{ A + t(B-A) \mid t \in \Rats \}. \] Let $L$ be the set of all such lines. Then the incidence structure $\Reals^3 = (P,L)$ is called \emph{Three-Space}.


\item[$\widehat{\mathcal{P}}$] \textbf{Dual Incidence Structures.} This example is a little different from the others in that it is not a single example, but rather a way to make new incidence structures out of old ones. Suppose we have an incidence structure $\mathcal{P} = (P,L)$. Given a point $x \in P$, we define $M_x = \{ \ell \in L \mid x \in \ell \}$. That is, $M_x$ is the set of all lines in $\mathcal{P}$ which contain the point $x$. If we let $M = \{ M_x \mid x \in P \}$, then certainly $\widehat{\mathcal{P}} = (L, M)$ is an incidence structure, which we call the \emph{dual} of $\mathcal{P}$.
\end{itemize}

\end{document}
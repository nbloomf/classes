\documentclass{article}
\usepackage{neb-macros}
\usepackage{tikz}
  \usetikzlibrary{calc,intersections}

\begin{document}

\CheapTitle{Euclidean Planes}

Recall that an incidence geometry is called \emph{euclidean} if, given any line $\ell$ and any point $p$ not on $\ell$, there is exactly one line passing through $p$ which is parallel to $\ell$. So far we have avoided using any assumptions about the uniqueness of parallel lines, and have been able to prove a good number of interesting results. We will now specialize to the Euclidean case for a while.

\begin{prop}[Converse of the Alternate Interior Angles Theorem]
In a Euclidean plane geometry, if two parallel lines are cut by a transversal, then alternate interior angles formed by the cut are congruent.
\end{prop}

\begin{proof}
(copy angle, use AIA, use uniqueness.)
\end{proof}

\end{document}

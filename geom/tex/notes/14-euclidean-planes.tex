\documentclass{article}
\usepackage{neb-macros}
\usepackage{tikz}
  \usetikzlibrary{calc,intersections}

\begin{document}

\CheapTitle{Euclidean Planes}

Recall that an incidence geometry is called \emph{euclidean} if, given any line $\ell$ and any point $p$ not on $\ell$, there is exactly one line passing through $p$ which is parallel to $\ell$. So far we have avoided using any assumptions about the uniqueness of parallel lines, and have been able to prove a good number of interesting results. We will now specialize to the Euclidean case for a while.

\begin{prop}[Converse of the Alternate Interior Angles Theorem]
In a Euclidean plane geometry, if two parallel lines are cut by a transversal, then alternate interior angles formed by the cut are congruent.
\end{prop}

\begin{proof}
(copy angle, use AIA, use uniqueness.)
\end{proof}

\begin{prop}
If $\ell$ and $m$ are parallel and $m$ and $t$ are parallel, then $\ell$ and $t$ are parallel.
\end{prop}

\begin{proof}
We can assume that $\ell$ and $t$ are distinct (if equal, they are parallel). Suppose BWOC that $\ell$ and $t$ meet at the unique point $x$. Since $\ell$ and $m$ are parallel, $x$ is not on $m$. By the Euclidean property, there is a unique line $s$ containing $x$ which is parallel to $m$. But both $\ell$ and $t$ satisfy this condition, and they are distinct - a contradiction.
\end{proof}

\begin{cor}
If $\ell_1$ and $\ell_2$ are parallel and $m$ is incident to $\ell_1$, then $m$ is incident to $\ell_2$.
\end{cor}

\begin{prop}
If $\ell_1$ and $m$ are perpendicular, and if $\ell_1$ and $\ell_2$ are parallel, then $\ell_2$ and $m$ are perpendicular.
\end{prop}

\begin{proof}
If $\ell_1 = \ell_2$ there's nothing to do. Otherwise $m$ is a transversal and the result follows from the converse of the AIA theorem.
\end{proof}

\begin{prop}
If $\ell_1$ and $\ell_2$ are parallel, $m_1$ and $\ell_1$ are perpendicular, and $m_2$ and $\ell_2$ are perpendicular, then $m_1$ and $m_2$ are parallel.
\end{prop}

\begin{proof}
$\ell_2$ and $m_1$ are perpendicular by the converse of AIA, and then $m_1$ and $m_2$ are parallel by AIA.
\end{proof}

\begin{construct}
Given 3 distinct noncollinear points $A$, $B$, and $C$, there is a unique circle which contains all of them. This circle is called the \emph{circumcircle} of $\Triangle{A}{B}{C}$, and its center is the \emph{circumcenter}.
\end{construct}

\begin{proof}
Let $\ell$ be the perpendicular bisector of $\Segment{A}{B}$ and let $m$ be the perpendicular bisector of $\Segment{B}{C}$. Now $\ell$ and $m$ must meet, since otherwise $\Line{A}{B}$ and $\Line{B}{C}$ are parallel (which they aren't, as they meet at the unique point $B$ (since $A$, $B$, and $C$ are not collinear)). Moreover they must meet at a \emph{unique} point, say $O$, since otherwise we can show that $A = C$. Recall that points $X$ on the perpendicular bisector of $\Segment{A}{B}$ have the property that $\Segment{A}{X} \equiv \Segment{B}{X}$. So we have $\Segment{A}{O} \equiv \Segment{B}{O} \equiv \Segment{C}{O}$, and thus $\Circle{O}{A}$ contains $A$, $B$, and $C$.
\end{proof}

\begin{prop} \mbox{}
\begin{enumerate}
\item Opposite angles of a parallelogram are congruent.
\item Opposite sides of a parallelogram are congruent.
\item The diagonals of a parallelogram bisect each other.
\end{enumerate}
\end{prop}

\begin{proof}
For the angles, use AIA and converse of AIA. For the sides, construct a diagonal and use AAS. For the diagonals, use converse of AIA and ASA.
\end{proof}

\begin{prop}[Thales' Theorem]
Suppose $A$ and $B$ are the opposite endpoints of a diameter of a circle centered at $O$, and that $C$ is a point on this circle distinct from $A$ and $B$. Then $\Angle{A}{C}{B}$ is right. Moreover, $\Angle{C}{A}{B}$ is congruent to the bisector of $\Angle{C}{O}{B}$.
\end{prop}

\begin{proof}
Construct the point $D$ on the intersection of $\Circle{O}{A}$ and $\Line{O}{C}$ by the circle separation axiom. Now $\Segment{A}{C} \equiv \Segment{B}{D}$ using SAS, and similarly $\Segment{C}{B} \equiv \Segment{A}{D}$. Now $\Triangle{A}{B}{C} \equiv \Triangle{B}{A}{D}$ by SSS, so that $\Angle{C}{B}{A} \equiv \Angle{D}{A}{B}$. Thus $\Line{B}{C}$ and $\Line{A}{D}$ are parallel by AIA. Now $\Triangle{B}{A}{C} \equiv \Triangle{D}{C}{A}$ by SSS, so that $\Angle{B}{C}{A} \equiv \Angle{D}{A}{C}$. Now $\Angle{D}{A}{C}$ and $\Angle{B}{C}{A}$ are supplementary by the converse of AIA. So $\Angle{B}{C}{A}$ is right.

Now let $M$ be the point on $\Segment{B}{C}$ such that $\Ray{O}{M}$ bisects $\Angle{C}{O}{B}$. (Use crossbar.) We have $\Angle{O}{C}{B} \equiv \Angle{O}{B}{C}$ by Pons Asinorum, so that $\Angle{O}{M}{C} \equiv \Angle{O}{M}{B}$ by ASA. Thus $\Angle{C}{M}{O}$ is right. By AIA, $\Line{O}{M}$ is parallel to $\Line{A}{C}$. By the converse of AIA, $\Angle{O}{C}{A} \equiv \Angle{C}{O}{M}$, and $\Angle{C}{A}{O} \equiv \Angle{C}{O}{M}$ by Pons Asinorum.
\end{proof}

\begin{prop}[Converse of Thales' Theorem]
Let $A$, $B$, and $C$ be distinct points. If $\Angle{A}{C}{B}$ is right, then $C$ is on the circle centered at the midpoint of $A$ and $B$ and passing through $A$.
\end{prop}

\begin{proof}
Let $M$ be the midpoint of $A$ and $B$, and copy $\Segment{M}{C}$ to the other side of $M$ at $D$ by circle separation. Now $\Segment{B}{C} \equiv \Segment{A}{D}$ by SAS, and similarly $\Segment{A}{C} \equiv \Segment{B}{D}$. So $\Triangle{A}{B}{C} \equiv \Triangle{B}{A}{D}$ by SSS. Now $\Line{B}{C}$ is parallel to $\Line{A}{D}$ by AIA, and so $\Angle{C}{A}{D} \equiv \Angle{A}{C}{B}$ using the converse of AIA. Now $\Triangle{C}{A}{D} \equiv \Triangle{A}{C}{B}$ by SAS, so that $\Segment{A}{B} \equiv \Segment{C}{D}$. Thus $\Segment{A}{M} \equiv \Segment{C}{M}$.
\end{proof}

(Here we used a lemma that if two segments are congruent, then their midsegments are congruent.)

\begin{construct}
Given a circle $\Circle{O}{A}$ and a point $B$ exterior to this circle, there exist two lines which are tangent to $\Circle{O}{A}$ and which pass through $B$.
\end{construct}

\begin{proof}
Construct the midpoint $M$ of $\Segment{B}{O}$, and construct circle centered at $M$ and passing through $O$. By the circle cut axiom, $\Circle{M}{O} \cap \Circle{O}{A}$ contains exactly two points, $X$ and $Y$. Note that $\Angle{O}{X}{B}$ and $\Angle{O}{Y}{B}$ are inscribed on the diameter of a circle, and thus are right; so $\Line{B}{X}$ and $\Line{B}{Y}$ are tangent to $\Circle{O}{A}$.

(still have to prove these are the only two.)
\end{proof}

\begin{prop}[Inscribed Angle Theorem]
Let $A$ and $B$ be distinct points on a circle centered at $O$. If $C$ is a point on $C$ such that $C$ and $O$ are on the same side of $\Line{A}{B}$, then $\Angle{A}{C}{B}$ is congruent to a bisector of $\Angle{A}{O}{B}$. In particular, any two such points form congruent angles.
\end{prop}

\subsection*{Altitudes and the Orthocenter}

\begin{dfn}
Let $A$, $B$, and $C$ be distinct noncollinear points, and let $F$ be the foot of $A$ on $\Line{B}{C}$. Then $\Segment{A}{F}$ is called an \emph{altitude} of $\Triangle{A}{B}{C}$. 
\end{dfn}

\begin{prop}[Orthocenter Theorem]
Let $A$, $B$, and $C$ be distinct noncollinear points. Then the lines containing the three altitudes of $\Triangle{A}{B}{C}$ are concurrent at a point $O$, called the \emph{orthocenter} of $\Triangle{A}{B}{C}$. 
\end{prop}

\begin{proof}

\end{proof}

\end{document}

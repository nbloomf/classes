\documentclass{article}
\usepackage{neb-macros}

\begin{document}

\CheapTitle{The Line Separation Property}

\begin{dfn}[Convexity]
Let $\mathcal{P}$ be an incidence geometry with a betweenness relation $\Between{\cdot}{\cdot}{\cdot}$. A non empty set $S$ of points in $\mathcal{P}$ is called \emph{convex} if whenever $x,y \in S$ are distinct points, $\Segment{x}{y} \subseteq S$.
\end{dfn}

\begin{dfn}[Line Separation Property]
Let $\mathcal{P}$ be an incidence geometry with a betweenness relation $\Between{\cdot}{\cdot}{\cdot}$. We say that this geometry has the \emph{Line Separation Property} if every line $\ell$ partitions the set of points not on $\ell$ into two nonempty, disjoint, convex sets, $H_1$ and $H_2$, with the property that if $x \in H_1$ and $y \in H_2$ then $\Segment{x}{y} \cap \ell = \{p\}$ for some point $p$. The sets $H_1$ and $H_2$ are called \emph{half-planes}.
\end{dfn}

\subsection*{Examples}

\subsection*{Ordered Geometries}

\begin{dfn}[Ordered Geometry]
Let $\mathcal{P}$ be an incidence geometry with a betweenness relation $\Between{\cdot}{\cdot}{\cdot}$. We say that $\mathcal{P}$ (with this betweenness relation) is an \emph{Ordered Geometry} if it has the Trichotomy Property, the 4-Point Property, the Interpolation Property, and the Line Separation Property.
\end{dfn}

\end{document}

\documentclass{article}
\usepackage{neb-titles}
\usepackage{flexfig}

\begin{document}

\ActivityTitle[class=Statistics, number=2, name={Probability}]

\begin{enumerate}
\item Suppose we roll a single 20-sided die, whose faces are numbered from 1 to 20. What is the probability that we roll a number strictly less than 5?



  
\vspace{5cm}

\item Suppose we draw a single card from a standard 52-card deck. What is the probability that we draw either a spade or a face card?

\textbf{Solution:} This is an event of the form $E \ \mathrm{or}\ F$, where $E$ is the event ``draw a spade'' and $F$ the event ``draw a face card''. There are 13 spades, 12 face cards, and 3 spades which are also face cards. Using the sum rule, we can say

\begin{eqnarray*}
P(E \ \mathrm{or}\ F) & = & P(E) + P(F) - P(E \ \mathrm{and}\ F) \\
 & = & \frac{13}{52} + \frac{12}{52} - \frac{3}{52} \\
 & = & \frac{11}{26}
\end{eqnarray*}


  
\vspace{5cm}

\item Suppose we roll two 6-sided dice, one pink and one green, whose faces are numbered from 1 to 6. What is the probability that we roll two numbers whose sum is exactly 10?

  
\vspace{5cm}

\item Suppose we roll a single 12-sided die with faces labeled 1 through 12.

\begin{enumerate*}
\item What is the sample space of this experiment?
\item Find the probabilities of the following events.
\begin{enumerate*}
\item Roll a 12
\item Roll a number divisible by 3
\item Roll a number greater than 8
\end{enumerate*}
\end{enumerate*}

  
\vspace{5cm}

\item Suppose we roll two 6-sided dice, one orange and one green, with faces labeled 1 through 6. Compute the probability of the following events.

\begin{enumerate*}
\item The dice show the same number.
\item The sum of the numbers on the dice is exactly 5.
\end{enumerate*}

  
\vspace{5cm}

\item Suppose we select a single card from a standard deck. Compute the probability of the following events.
\begin{enumerate*}
\item The card is a 7.
\item The card is black.
\item The card is a diamond.
\item The card is a face card (Jack, Queen, or King).
\end{enumerate*}

  
\vspace{5cm}

\item Suppose we flip a coin four times in a row, to get a sequence of coin flips. For example, if we flip heads, then tails, then heads, then heads, the result is $(H,T,H,H)$. Write down the sample space for this experiment. Then compute the probability of the following events.
\begin{enumerate*}
\item We flip tails four times.
\item We flip exactly two heads.
\item We flip at least three tails.
\item The first two flips are tails.
\end{enumerate*}

  
\vspace{5cm}

\item Suppose we draw a single card from a standard deck. What is the probability that the card is either a heart or a face card?

  
\vspace{5cm}

\item Two cards are drawn from a standard deck without replacement. What is the probability that both have the same face value? (E.g. both are aces.)

  
\vspace{5cm}

\item Three cards are drawn from a standard deck without replacement. What is the probability that all three have the same face value? (E.g. all three are aces.)

  
\vspace{5cm}
\end{enumerate}

\end{document}

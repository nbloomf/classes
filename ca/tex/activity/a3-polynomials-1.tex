\documentclass{article}
\usepackage{neb-titles}
\usepackage{flexfig}

\pagestyle{empty}

\begin{document}

\ActivityTitle[class={College Algebra}, number={3}, name={Polynomials 1}]

\begin{enumerate}
\item Fill in the boxes to describe the long-term behavior of the following polynomial. \[ p(x) = 3x^3 - 2x + 1 \].

\begin{itemize}
\item As $x \rightarrow \infty$, $p(x) \rightarrow$ \framebox(30,20){} \vspace{0.5cm}
\item As $x \rightarrow -\infty$, $p(x) \rightarrow$ \framebox(30,20){}
\end{itemize} \vspace{1cm}

\item Using polynomial long division, find the quotient and remainder when \[ a(x) = x^5 - 2x^4 - 2x^3 + 8x^2 - 7x + 2 \] is divided by \[ b(x) = x^3 - 3x + 2. \] \vspace{6cm}

\item Use synthetic division to find the quotient and remainder when \[ a(x) = x^5 - x^4 - 5x^3 + 5x^2 + 4x - 4 \] is divided by $b(x) = x + 2$. \vspace{3cm}

\newpage

\item The polynomial \[ p(x) = x^5 - 4x^4 - 10x^3 + 40x^2 + 9x - 36 \] has roots at ${4;1;-3;3}$. Completely factor $p(x)$ as a product of linear factors. \vspace{7cm}

\item The polynomial \[ p(x) = x^5 - 7x^4 + 19x^3 - 25x^2 + 16x - 4 \] has roots at 1 and 2. Find the multiplicity of these roots. \vspace{7cm}

\item Construct a polynomial which has roots at -2, 1, and 2. \vspace{7cm}
\end{enumerate}

\end{document}

\documentclass{article}
\usepackage{neb-titles}
\usepackage{neb-macros}
\usepackage{flexfig}

\begin{document}

\ActivityTitle[class=Calculus I, number=2, name=Continuity (Solutions)]

\begin{enumerate}
\item Compute the limit of the difference quotient \[ \lim_{x \rightarrow t} \frac{f(x) - f(t)}{x - t} \] when $f(x) = 9 x + 12$ and $t = 7$.

\textbf{Solution:} We have
\begin{eqnarray*}
\lim_{x \rightarrow 7} \frac{f(x) - f(7)}{x - 7} & = & \lim_{x \rightarrow 7} \frac{(9 x + 12) - (9 \cdot 7 + 12)}{x - 7} \\
 & = & \lim_{x \rightarrow 7} \frac{9 x - 9 \cdot 7}{x - 7} \\
 & = & \lim_{x \rightarrow 7} \frac{9(x - 7)}{x - 7} \\
 & = & \lim_{x \rightarrow 7} 9 \\
 & = & 9 \\
\end{eqnarray*}


  

\item Compute the limit of the difference quotient \[ \lim_{x \rightarrow t} \frac{f(x) - f(t)}{x - t} \] when $f(x) = 12 x^2 + 11 x + 9$ and $t = 7$.

\textbf{Solution:} We have
\begin{eqnarray*}
\lim_{x \rightarrow 7} \frac{f(x) - f(7)}{x - 7} & = & \lim_{x \rightarrow 7} \frac{(12 x^2 + 11 x + 9) - (12 (7)^2 + 11(7) + 9)}{x - 7} \\
 & = & \lim_{x \rightarrow 7} \frac{12 x^2 - 12(7)^2 + 11 x - 11(7) + 9 - 9}{x - 7} \\
 & = & \lim_{x \rightarrow 7} \frac{12(x^2 - 7^2) + 11(x - 7)}{x - 7} \\
 & = & \lim_{x \rightarrow 7} \frac{12(x - 7)(x + 7) + 11(x - 7)}{x - 7} \\
 & = & \lim_{x \rightarrow 7} \frac{(x - 7)(12(x + 7) + 11)}{x - 7} \\
 & = & \lim_{x \rightarrow 7} \left( 12(x + 7) + 11 \right) \\
 & = & 12(7 + 7) + 11\\
 & = & 179
\end{eqnarray*}


  

\item Compute the limit of the difference quotient \[ \lim_{x \rightarrow t} \frac{f(x) - f(t)}{x - t} \] when $f(x) = \sqrt{x + 2}$ and $t = 1$.

\textbf{Solution:} We have
\begin{eqnarray*}
\lim_{x \rightarrow 1} \frac{f(x) - f(1)}{x - 1} & = & \lim_{x \rightarrow 1} \frac{\sqrt{x + 2} - \sqrt{3}}{x - 1} \\
 & = & \lim_{x \rightarrow 1} \left( \frac{\sqrt{x + 2} - \sqrt{3}}{x - 1} \cdot \frac{\sqrt{x + 2} + \sqrt{3}}{\sqrt{x + 2} + \sqrt{3}} \right) \\
 & = & \lim_{x \rightarrow 1} \frac{(x + 2) - (3)}{(x - 1)\left( \sqrt{x + 2} + \sqrt{3} \right)} \\
 & = & \lim_{x \rightarrow 1} \frac{x - 1}{(x - 1)\left( \sqrt{x + 2} + \sqrt{3} \right)} \\
 & = & \lim_{x \rightarrow 1} \frac{1}{\sqrt{x + 2} + \sqrt{3}} \\
 & = & \frac{1}{2\sqrt{3}} \\
\end{eqnarray*}


  

\item Compute the following limit. \[ \lim_{x \rightarrow 0} \frac{\sin(2 x)}{x} \]

\textbf{Solution:} Note that
\begin{eqnarray*}
\lim_{x \rightarrow 0} \frac{\sin(2 x)}{x}
 & = & \lim_{x \rightarrow 0} \frac{k\sin(2 x)}{2 x} \\
 & = & 2 \lim_{x \rightarrow 0} \frac{\sin(2 x)}{2 x} \\
 & = & 2 \lim_{x \rightarrow 0} \sinc(2 x) \\
 & = & 2 \sinc\left( \lim_{x \rightarrow 0} 2 x \right) \quad \mathrm{(since}\ \sinc\ \mathrm{is\ continuous)} \\
 & = & 2 \sinc(0) \\
 & = & 2.
\end{eqnarray*}


  

\item Compute the following limit. \[ \lim_{x \rightarrow 0} \frac{7 x^2 + 8 x + \sin x}{x} \]

\textbf{Solution:} This expression is not defined if $x = 0$. However, we can split this fraction like so:

\begin{eqnarray*}
\lim_{x \rightarrow 0} \frac{7 x^2 + 8 x + \sin(x)}{x}
 & = & \lim_{x \rightarrow 0} \left( \frac{7 x^2 + 8 x}{x} + \frac{\sin x}{x} \right) \\
 & = & \lim_{x \rightarrow 0} \left( 7 x + 8 + \frac{\sin x}{x} \right). \\
\end{eqnarray*}

Recall that the limit of a sum is the sum of limits, \emph{provided} the limit of each summand exists. In this case they do, and we have

\begin{eqnarray*}
 & = & \lim_{x \rightarrow 0} \left( 7 x + 8 \right) + \lim_{x \rightarrow 0} \frac{\sin x}{x} \\
 & = & 8 + 1 \\
 & = & 9. \\
\end{eqnarray*}


  

\item Let $f(x)$ be the function \[ f(x) = \left\{ \begin{array}{ll} \frac{x-b}{b + 3} & \mathrm{if}\ x < 0 \\ & \\ x^2 + b & \mathrm{if}\ x \geq 0. \end{array}\right. \] Find the value(s) of the constant $b$ such that $f(x)$ is continuous everywhere.

\textbf{Solution:} Remember that $\dlim_{x \rightarrow 0} f(x)$ exists precisely when the one-sided limits $\dlim_{x \rightarrow 0^+} f(x)$ and $\dlim_{x \rightarrow 0^-} f(x)$ exist and are equal to one another. In this case, we have \[ \lim_{x \rightarrow 0^+} f(x) = \lim_{x \rightarrow 0^+} (x^2 + b) = b \] and \[ \lim_{x \rightarrow 0^-} f(x) = \lim_{x \rightarrow 0^+} \frac{x-b}{b + 3} = \frac{-b}{b + 3}. \]
Setting these equal, we have \[ b = \frac{-b}{b + 3}. \]
The values for $b$ we want are precisely the solutions of this equation.
Clearing denominators, we have \[ b^2 + 3 b = -b, \] and solving for zero we have \[ b^2 + 4b = 0 \quad\quad \mathrm{which\ factors\ as}\ \quad\quad b (b + 4) = 0. \]
So $b = 0$ or $b = -4$.

  
\end{enumerate}

\end{document}
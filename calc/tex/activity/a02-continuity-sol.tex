\documentclass{article}
\usepackage{neb-titles}
\usepackage{neb-macros}
\usepackage{flexfig}

\begin{document}

\ActivityTitle[class=Calculus I, number=2, name=Continuity (Solutions)]

\begin{enumerate}
\item Compute the limit of the difference quotient \[ \lim_{x \rightarrow t} \frac{f(x) - f(t)}{x - t} \] when $f(x) = 5 x + 3$ and $t = 3$.

\textbf{Solution:} We have
\begin{eqnarray*}
\lim_{x \rightarrow 3} \frac{f(x) - f(3)}{x - 3} & = & \lim_{x \rightarrow 3} \frac{(5 x + 3) - (5 \cdot 3 + 3)}{x - 3} \\
 & = & \lim_{x \rightarrow 3} \frac{5 x - 5 \cdot 3}{x - 3} \\
 & = & \lim_{x \rightarrow 3} \frac{5(x - 3)}{x - 3} \\
 & = & \lim_{x \rightarrow 3} 5 \\
 & = & 5 \\
\end{eqnarray*}


  
\vspace{1cm}

\item Compute the limit of the difference quotient \[ \lim_{x \rightarrow t} \frac{f(x) - f(t)}{x - t} \] when $f(x) = 11 x^2 + 11 x + 5$ and $t = 3$.

\textbf{Solution:} We have
\begin{eqnarray*}
\lim_{x \rightarrow 3} \frac{f(x) - f(3)}{x - 3} & = & \lim_{x \rightarrow 3} \frac{(11 x^2 + 11 x + 5) - (11 (3)^2 + 11(3) + 5)}{x - 3} \\
 & = & \lim_{x \rightarrow 3} \frac{11 x^2 - 11(3)^2 + 11 x - 11(3) + 5 - 5}{x - 3} \\
 & = & \lim_{x \rightarrow 3} \frac{11(x^2 - 3^2) + 11(x - 3)}{x - 3} \\
 & = & \lim_{x \rightarrow 3} \frac{11(x - 3)(x + 3) + 11(x - 3)}{x - 3} \\
 & = & \lim_{x \rightarrow 3} \frac{(x - 3)(11(x + 3) + 11)}{x - 3} \\
 & = & \lim_{x \rightarrow 3} \left( 11(x + 3) + 11 \right) \\
 & = & 11(3 + 3) + 11\\
 & = & 77
\end{eqnarray*}


  
\vspace{1cm}

\item Compute the limit of the difference quotient \[ \lim_{x \rightarrow t} \frac{f(x) - f(t)}{x - t} \] when $f(x) = \sqrt{x + 3}$ and $t = 2$.

\textbf{Solution:} We have
\begin{eqnarray*}
\lim_{x \rightarrow 2} \frac{f(x) - f(2)}{x - 2} & = & \lim_{x \rightarrow 2} \frac{\sqrt{x + 3} - \sqrt{5}}{x - 2} \\
 & = & \lim_{x \rightarrow 2} \left( \frac{\sqrt{x + 3} - \sqrt{5}}{x - 2} \cdot \frac{\sqrt{x + 3} + \sqrt{5}}{\sqrt{x + 3} + \sqrt{5}} \right) \\
 & = & \lim_{x \rightarrow 2} \frac{(x + 3) - (5)}{(x - 2)\left( \sqrt{x + 3} + \sqrt{5} \right)} \\
 & = & \lim_{x \rightarrow 2} \frac{x - 2}{(x - 2)\left( \sqrt{x + 3} + \sqrt{5} \right)} \\
 & = & \lim_{x \rightarrow 2} \frac{1}{\sqrt{x + 3} + \sqrt{5}} \\
 & = & \frac{1}{2\sqrt{5}} \\
\end{eqnarray*}


  
\vspace{1cm}

\item Compute the following limit. \[ \lim_{x \rightarrow 0} \frac{\sin(5 x)}{x} \]

\textbf{Solution:} Note that
\begin{eqnarray*}
\lim_{x \rightarrow 0} \frac{\sin(5 x)}{x}
 & = & \lim_{x \rightarrow 0} \frac{5\sin(5 x)}{5 x} \\
 & = & 5 \lim_{x \rightarrow 0} \frac{\sin(5 x)}{5 x} \\
 & = & 5 \lim_{x \rightarrow 0} \sinc(5 x) \\
 & = & 5 \sinc\left( \lim_{x \rightarrow 0} 5 x \right) \quad \mathrm{(since}\ \sinc\ \mathrm{is\ continuous)} \\
 & = & 5 \sinc(0) \\
 & = & 5.
\end{eqnarray*}


  
\vspace{1cm}

\item Compute the following limit. \[ \lim_{x \rightarrow 0} \frac{12 x^2 + 12 x + \sin x}{x} \]

\textbf{Solution:} This expression is not defined if $x = 0$. However, we can split this fraction like so:

\begin{eqnarray*}
\lim_{x \rightarrow 0} \frac{12 x^2 + 12 x + \sin(x)}{x}
 & = & \lim_{x \rightarrow 0} \left( \frac{12 x^2 + 12 x}{x} + \frac{\sin x}{x} \right) \\
 & = & \lim_{x \rightarrow 0} \left( 12 x + 12 + \frac{\sin x}{x} \right). \\
\end{eqnarray*}

Recall that the limit of a sum is the sum of limits, \emph{provided} the limit of each summand exists. In this case they do, and we have

\begin{eqnarray*}
 & = & \lim_{x \rightarrow 0} \left( 12 x + 12 \right) + \lim_{x \rightarrow 0} \frac{\sin x}{x} \\
 & = & 12 + 1 \\
 & = & 13. \\
\end{eqnarray*}


  
\vspace{1cm}

\item Let $f(x)$ be the function \[ f(x) = \left\{ \begin{array}{ll} \frac{x-b}{b + 6} & \mathrm{if}\ x < 0 \\ & \\ x^2 + b & \mathrm{if}\ x \geq 0. \end{array}\right. \] Find the value(s) of the constant $b$ such that $f(x)$ is continuous everywhere.

\textbf{Solution:} Remember that $\dlim_{x \rightarrow 0} f(x)$ exists precisely when the one-sided limits $\dlim_{x \rightarrow 0^+} f(x)$ and $\dlim_{x \rightarrow 0^-} f(x)$ exist and are equal to one another. In this case, we have \[ \lim_{x \rightarrow 0^+} f(x) = \lim_{x \rightarrow 0^+} (x^2 + b) = b \] and \[ \lim_{x \rightarrow 0^-} f(x) = \lim_{x \rightarrow 0^+} \frac{x-b}{b + 6} = \frac{-b}{b + 6}. \]
Setting these equal, we have \[ b = \frac{-b}{b + 6}. \]
The values for $b$ we want are precisely the solutions of this equation.
Clearing denominators, we have \[ b^2 + 6 b = -b, \] and solving for zero we have \[ b^2 + 7b = 0 \quad\quad \mathrm{which\ factors\ as}\ \quad\quad b (b + 7) = 0. \]
So $b = 0$ or $b = -7$.

  
\vspace{1cm}
\end{enumerate}

\end{document}
\documentclass{article}
\usepackage{neb-titles}
\usepackage{neb-macros}
\usepackage{flexfig}

\begin{document}

\ActivityTitle[class=Calculus I, number=2, name=Continuity (Solutions)]

\begin{enumerate}
\item Compute the limit of the difference quotient \[ \lim_{x \rightarrow t} \frac{f(x) - f(t)}{x - t} \] when $f(x) = 12 x + 10$ and $t = 10$.

\textbf{Solution:} We have
\begin{eqnarray*}
\lim_{x \rightarrow 10} \frac{f(x) - f(10)}{x - 10} & = & \lim_{x \rightarrow 10} \frac{(12 x + 10) - (12 \cdot 10 + 10)}{x - 10} \\
 & = & \lim_{x \rightarrow 10} \frac{12 x - 12 \cdot 10}{x - 10} \\
 & = & \lim_{x \rightarrow 10} \frac{12(x - 10)}{x - 10} \\
 & = & \lim_{x \rightarrow 10} 12 \\
 & = & 12 \\
\end{eqnarray*}


  
\vspace{1cm}

\item Compute the limit of the difference quotient \[ \lim_{x \rightarrow t} \frac{f(x) - f(t)}{x - t} \] when $f(x) = 11 x^2 + 4 x + 1$ and $t = 9$.

\textbf{Solution:} We have
\begin{eqnarray*}
\lim_{x \rightarrow 9} \frac{f(x) - f(9)}{x - 9} & = & \lim_{x \rightarrow 9} \frac{(11 x^2 + 4 x + 1) - (11 (9)^2 + 4(9) + 1)}{x - 9} \\
 & = & \lim_{x \rightarrow 9} \frac{11 x^2 - 11(9)^2 + 4 x - 4(9) + 1 - 1}{x - 9} \\
 & = & \lim_{x \rightarrow 9} \frac{11(x^2 - 9^2) + 4(x - 9)}{x - 9} \\
 & = & \lim_{x \rightarrow 9} \frac{11(x - 9)(x + 9) + 4(x - 9)}{x - 9} \\
 & = & \lim_{x \rightarrow 9} \frac{(x - 9)(11(x + 9) + 4)}{x - 9} \\
 & = & \lim_{x \rightarrow 9} \left( 11(x + 9) + 4 \right) \\
 & = & 11(9 + 9) + 4\\
 & = & 202
\end{eqnarray*}


  
\vspace{1cm}

\item Compute the limit of the difference quotient \[ \lim_{x \rightarrow t} \frac{f(x) - f(t)}{x - t} \] when $f(x) = \sqrt{x + 6}$ and $t = 4$.

\textbf{Solution:} We have
\begin{eqnarray*}
\lim_{x \rightarrow 4} \frac{f(x) - f(4)}{x - 4} & = & \lim_{x \rightarrow 4} \frac{\sqrt{x + 6} - \sqrt{10}}{x - 4} \\
 & = & \lim_{x \rightarrow 4} \left( \frac{\sqrt{x + 6} - \sqrt{10}}{x - 4} \cdot \frac{\sqrt{x + 6} + \sqrt{10}}{\sqrt{x + 6} + \sqrt{10}} \right) \\
 & = & \lim_{x \rightarrow 4} \frac{(x + 6) - (10)}{(x - 4)\left( \sqrt{x + 6} + \sqrt{10} \right)} \\
 & = & \lim_{x \rightarrow 4} \frac{x - 4}{(x - 4)\left( \sqrt{x + 6} + \sqrt{10} \right)} \\
 & = & \lim_{x \rightarrow 4} \frac{1}{\sqrt{x + 6} + \sqrt{10}} \\
 & = & \frac{1}{2\sqrt{10}} \\
\end{eqnarray*}


  
\vspace{1cm}

\item Compute the following limit. \[ \lim_{x \rightarrow 0} \frac{\sin(9 x)}{x} \]

\textbf{Solution:} Note that
\begin{eqnarray*}
\lim_{x \rightarrow 0} \frac{\sin(9 x)}{x}
 & = & \lim_{x \rightarrow 0} \frac{9\sin(9 x)}{9 x} \\
 & = & 9 \lim_{x \rightarrow 0} \frac{\sin(9 x)}{9 x} \\
 & = & 9 \lim_{x \rightarrow 0} \sinc(9 x) \\
 & = & 9 \sinc\left( \lim_{x \rightarrow 0} 9 x \right) \quad \mathrm{(since}\ \sinc\ \mathrm{is\ continuous)} \\
 & = & 9 \sinc(0) \\
 & = & 9.
\end{eqnarray*}


  
\vspace{1cm}

\item Compute the following limit. \[ \lim_{x \rightarrow 0} \frac{8 x^2 + 2 x + \sin x}{x} \]

\textbf{Solution:} This expression is not defined if $x = 0$. However, we can split this fraction like so:

\begin{eqnarray*}
\lim_{x \rightarrow 0} \frac{8 x^2 + 2 x + \sin(x)}{x}
 & = & \lim_{x \rightarrow 0} \left( \frac{8 x^2 + 2 x}{x} + \frac{\sin x}{x} \right) \\
 & = & \lim_{x \rightarrow 0} \left( 8 x + 2 + \frac{\sin x}{x} \right). \\
\end{eqnarray*}

Recall that the limit of a sum is the sum of limits, \emph{provided} the limit of each summand exists. In this case they do, and we have

\begin{eqnarray*}
 & = & \lim_{x \rightarrow 0} \left( 8 x + 2 \right) + \lim_{x \rightarrow 0} \frac{\sin x}{x} \\
 & = & 2 + 1 \\
 & = & 3. \\
\end{eqnarray*}


  
\vspace{1cm}

\item Let $f(x)$ be the function \[ f(x) = \left\{ \begin{array}{ll} \frac{x-b}{b + 9} & \mathrm{if}\ x < 0 \\ & \\ x^2 + b & \mathrm{if}\ x \geq 0. \end{array}\right. \] Find the value(s) of the constant $b$ such that $f(x)$ is continuous everywhere.

\textbf{Solution:} Remember that $\dlim_{x \rightarrow 0} f(x)$ exists precisely when the one-sided limits $\dlim_{x \rightarrow 0^+} f(x)$ and $\dlim_{x \rightarrow 0^-} f(x)$ exist and are equal to one another. In this case, we have \[ \lim_{x \rightarrow 0^+} f(x) = \lim_{x \rightarrow 0^+} (x^2 + b) = b \] and \[ \lim_{x \rightarrow 0^-} f(x) = \lim_{x \rightarrow 0^+} \frac{x-b}{b + 9} = \frac{-b}{b + 9}. \]
Setting these equal, we have \[ b = \frac{-b}{b + 9}. \]
The values for $b$ we want are precisely the solutions of this equation.
Clearing denominators, we have \[ b^2 + 9 b = -b, \] and solving for zero we have \[ b^2 + 10b = 0 \quad\quad \mathrm{which\ factors\ as}\ \quad\quad b (b + 10) = 0. \]
So $b = 0$ or $b = -10$.

  
\vspace{1cm}
\end{enumerate}

\end{document}
\documentclass{article}
\usepackage{neb-titles}
\usepackage{neb-macros}
\usepackage{flexfig}

\begin{document}

\ActivityTitle[class=Calculus I, number=1, name=Limits (Solutions)]

\begin{enumerate}
\item Compute the following limit. \[ \lim_{x \rightarrow 2} \left( 2 x^2 + 8 x + 9 \right) \]

\textbf{Solution:} Since this expression is a polynomial, we can find the limit as $x$ approaches 2 by evaluating at 2.

\begin{eqnarray*}
\lim_{x \rightarrow 2} \left( 2 x^2 + 8 x + 9 \right)
 & = & 2 (2)^2 + 8 (2) + 9 \\
 & = & 8 + 16 + 9 \\
 & = & 33.
\end{eqnarray*}


  
\vspace{1cm}

\item Compute the following limit. \[ \displaystyle\lim_{x \rightarrow 9} \frac{x^2 - 10x + 9}{x - 9} \]

\textbf{Solution:} Note that this rational function is not defined at $x = 9$. However, factoring the numerator we have 

\begin{eqnarray*}
\lim_{x \rightarrow 9} \frac{x^2 - 10x + 9}{x - 9}
 & = & \lim_{x \rightarrow 9} \frac{(x - 9)(x - 1)}{x - 9} \\
 & = & \lim_{x \rightarrow 9} \frac{\cancel{(x - 9)}(x - 1)}{\cancel{x - 9}} \\
 & = & \lim_{x \rightarrow 9} (x - 1) \\
 & = & 8. \\
\end{eqnarray*}


  
\vspace{1cm}

\item Compute the following limit. \[ \lim_{x \rightarrow 8} \frac{x^2 - 64}{x - 8} \]

\textbf{Solution:} Note that this expression is not defined if $x = 8$. However, we can factor the numerator as a difference of squares and cancel.

\[ \lim_{x \rightarrow 8} \frac{x^2 - 64}{x - 8} = \lim_{x \rightarrow 8} \frac{(x - 8)(x + 8)}{x - 8} = \lim_{x \rightarrow 8} (x + 8) = 16 \]


  
\vspace{1cm}

\item Compute the following limit. \[ \lim_{x \rightarrow 25} \frac{x - 25}{\sqrt{x} - 5} \]

\textbf{Solution:} Note that this expression is not defined if $x = 25$. However, we can factor the numerator as a difference of squares.

\[ \lim_{x \rightarrow 25} \frac{x - 25}{\sqrt{x} - 5} = \lim_{x \rightarrow 25} \frac{(\sqrt{x} - 5)(\sqrt{x} + 5)}{\sqrt{x} - 5} = \lim_{x \rightarrow 25} (\sqrt{x} + 5) = 10 \]


  
\vspace{1cm}

\item Compute the following limit. \[ \lim_{x \rightarrow 18} \frac{\sqrt{x - 2} - 4}{x - 18} \]

\textbf{Solution:} Note that this expression is not defined if $x = 18$. But also note that if we multiply the numerator by its radical conjugate, something nice happens:
\[ \left(\sqrt{x - 2} - 4\right)\left(\sqrt{x - 2} + 4\right) = x - 2 - 16 = x - 18. \]
Let's try multiplying by 1, but write 1 as $\sqrt{x - 2} + 4$ over itself.
\begin{eqnarray*}
\lim_{x \rightarrow 18} \frac{\sqrt{x - 2} - 4}{x - 18} & = & \lim_{x \rightarrow 18} \left( \frac{\sqrt{x - 2} - 4}{x - 18} \cdot \frac{\sqrt{x - 2} + 4}{\sqrt{x - 2} + 4} \right) \\
 & = & \lim_{x \rightarrow 18} \frac{x - 18}{(x - 18)(\sqrt{x - 2} + 4)} \\
 & = & \lim_{x \rightarrow 18} \frac{1}{\sqrt{x - 2} + 4} \\
 & = & \frac{1}{8}
\end{eqnarray*}


  
\vspace{1cm}

\item Compute the following limit. \[ \lim_{x \rightarrow 33} \frac{\sqrt{x - 8} - 5}{x - 33} \]

\textbf{Solution:} Note that this expression is not defined if $x = 33$. But also note that if we multiply the numerator by its radical conjugate, something nice happens:
\[ \left(\sqrt{x - 8} - 5\right)\left(\sqrt{x - 8} + 5\right) = x - 8 - 25 = x - 33. \]
Let's try multiplying by 1, but write 1 as $\sqrt{x - 8} + 5$ over itself.
\begin{eqnarray*}
\lim_{x \rightarrow 33} \frac{\sqrt{x - 8} - 5}{x - 33} & = & \lim_{x \rightarrow 33} \left( \frac{\sqrt{x - 8} - 5}{x - 33} \cdot \frac{\sqrt{x - 8} + 5}{\sqrt{x - 8} + 5} \right) \\
 & = & \lim_{x \rightarrow 33} \frac{x - 33}{(x - 33)(\sqrt{x - 8} + 5)} \\
 & = & \lim_{x \rightarrow 33} \frac{1}{\sqrt{x - 8} + 5} \\
 & = & \frac{1}{10}
\end{eqnarray*}


  
\vspace{1cm}

\item Compute the following limit. \[ \lim_{x \rightarrow -2} \frac{x^3 + 2x^2 - x - 2}{x + 2} \]

\textbf{Solution:} Note that this rational function is not defined if $x = -2$. However, $-2$ is a root of both the numerator and the denominator, so we can factor (either by grouping or using long or synthetic division) and cancel.
\begin{eqnarray*}
\lim_{x \rightarrow -2} \frac{x^3 + 2x^2 - x - 2}{x + 2} & = & \lim_{x \rightarrow -2} \frac{x^2(x + 2) - 1(x + 2)}{x + 2} \\
 & = & \lim_{x \rightarrow -2} \frac{(x + 2)(x^2 - 1)}{x + 2} \\
 & = & \lim_{x \rightarrow -2} (x^2 - 1) \\
 & = & 3
\end{eqnarray*}


  
\vspace{1cm}

\item Compute the following limit. \[ \lim_{x \rightarrow -2} \frac{x^3 + 4x^2 + x - 6}{x + 2} \]

\textbf{Solution:} Note that this rational function is not defined if $x = -2$. However, $-2$ is a root of both the numerator and the denominator, so we can factor (using either long or synthetic division) and cancel.
\[ \lim_{x \rightarrow -2} \frac{x^3 + 4x^2 + x - 6}{x + 2} = \lim_{x \rightarrow -2} \frac{(x + 2)(x^2 + 2x - 3)}{x + 2} = \lim_{x \rightarrow -2} (x^2 + 2x - 3) = -3 \]


  
\vspace{1cm}
\end{enumerate}

\end{document}

\documentclass{article}
\usepackage{neb-titles}
\usepackage{neb-macros}
\usepackage{flexfig}

\begin{document}

\ActivityTitle[class=Calculus I, number=1, name=Limits (Solutions)]

\begin{enumerate}
\item Compute the following limit. \[ \lim_{x \rightarrow 5} \left( 3 x^2 + 2 x + 8 \right) \]

\textbf{Solution:} Since this expression is a polynomial, we can find the limit as $x$ approaches 5 by evaluating at 5.

\begin{eqnarray*}
\lim_{x \rightarrow 5} \left( 3 x^2 + 2 x + 8 \right)
 & = & 3 (5)^2 + 2 (5) + 8 \\
 & = & 75 + 10 + 8 \\
 & = & 93.
\end{eqnarray*}


  
\vspace{1cm}

\item Compute the following limit. \[ \displaystyle\lim_{x \rightarrow 6} \frac{x^2 - 2x - 24}{x - 6} \]

\textbf{Solution:} Note that this rational function is not defined at $x = 6$. However, factoring the numerator we have 

\begin{eqnarray*}
\lim_{x \rightarrow 6} \frac{x^2 - 2x - 24}{x - 6}
 & = & \lim_{x \rightarrow 6} \frac{(x - 6)(x + 4)}{x - 6} \\
 & = & \lim_{x \rightarrow 6} \frac{\cancel{(x - 6)}(x + 4)}{\cancel{x - 6}} \\
 & = & \lim_{x \rightarrow 6} (x + 4) \\
 & = & 10. \\
\end{eqnarray*}


  
\vspace{1cm}

\item Compute the following limit. \[ \lim_{x \rightarrow 2} \frac{x^2 - 4}{x - 2} \]

\textbf{Solution:} Note that this expression is not defined if $x = 2$. However, we can factor the numerator as a difference of squares and cancel.

\[ \lim_{x \rightarrow 2} \frac{x^2 - 4}{x - 2} = \lim_{x \rightarrow 2} \frac{(x - 2)(x + 2)}{x - 2} = \lim_{x \rightarrow 2} (x + 2) = 4 \]


  
\vspace{1cm}

\item Compute the following limit. \[ \lim_{x \rightarrow 64} \frac{x - 64}{\sqrt{x} - 8} \]

\textbf{Solution:} Note that this expression is not defined if $x = 64$. However, we can factor the numerator as a difference of squares.

\[ \lim_{x \rightarrow 64} \frac{x - 64}{\sqrt{x} - 8} = \lim_{x \rightarrow 64} \frac{(\sqrt{x} - 8)(\sqrt{x} + 8)}{\sqrt{x} - 8} = \lim_{x \rightarrow 64} (\sqrt{x} + 8) = 16 \]


  
\vspace{1cm}

\item Compute the following limit. \[ \lim_{x \rightarrow 14} \frac{\sqrt{x - 5} - 3}{x - 14} \]

\textbf{Solution:} Note that this expression is not defined if $x = 14$. But also note that if we multiply the numerator by its radical conjugate, something nice happens:
\[ \left(\sqrt{x - 5} - 3\right)\left(\sqrt{x - 5} + 3\right) = x - 5 - 9 = x - 14. \]
Let's try multiplying by 1, but write 1 as $\sqrt{x - 5} + 3$ over itself.
\begin{eqnarray*}
\lim_{x \rightarrow 14} \frac{\sqrt{x - 5} - 3}{x - 14} & = & \lim_{x \rightarrow 14} \left( \frac{\sqrt{x - 5} - 3}{x - 14} \cdot \frac{\sqrt{x - 5} + 3}{\sqrt{x - 5} + 3} \right) \\
 & = & \lim_{x \rightarrow 14} \frac{x - 14}{(x - 14)(\sqrt{x - 5} + 3)} \\
 & = & \lim_{x \rightarrow 14} \frac{1}{\sqrt{x - 5} + 3} \\
 & = & \frac{1}{6}
\end{eqnarray*}


  
\vspace{1cm}

\item Compute the following limit. \[ \lim_{x \rightarrow 8} \frac{\sqrt{x - 7} - 1}{x - 8} \]

\textbf{Solution:} Note that this expression is not defined if $x = 8$. But also note that if we multiply the numerator by its radical conjugate, something nice happens:
\[ \left(\sqrt{x - 7} - 1\right)\left(\sqrt{x - 7} + 1\right) = x - 7 - 1 = x - 8. \]
Let's try multiplying by 1, but write 1 as $\sqrt{x - 7} + 1$ over itself.
\begin{eqnarray*}
\lim_{x \rightarrow 8} \frac{\sqrt{x - 7} - 1}{x - 8} & = & \lim_{x \rightarrow 8} \left( \frac{\sqrt{x - 7} - 1}{x - 8} \cdot \frac{\sqrt{x - 7} + 1}{\sqrt{x - 7} + 1} \right) \\
 & = & \lim_{x \rightarrow 8} \frac{x - 8}{(x - 8)(\sqrt{x - 7} + 1)} \\
 & = & \lim_{x \rightarrow 8} \frac{1}{\sqrt{x - 7} + 1} \\
 & = & \frac{1}{2}
\end{eqnarray*}


  
\vspace{1cm}

\item Compute the following limit. \[ \lim_{x \rightarrow 3} \frac{x^3 - 3x^2 - 4x + 12}{x - 3} \]

\textbf{Solution:} Note that this rational function is not defined if $x = 3$. However, $3$ is a root of both the numerator and the denominator, so we can factor (either by grouping or using long or synthetic division) and cancel.
\begin{eqnarray*}
\lim_{x \rightarrow 3} \frac{x^3 - 3x^2 - 4x + 12}{x - 3} & = & \lim_{x \rightarrow 3} \frac{x^2(x - 3) - 4(x - 3)}{x - 3} \\
 & = & \lim_{x \rightarrow 3} \frac{(x - 3)(x^2 - 4)}{x - 3} \\
 & = & \lim_{x \rightarrow 3} (x^2 - 4) \\
 & = & 5
\end{eqnarray*}


  
\vspace{1cm}

\item Compute the following limit. \[ \lim_{x \rightarrow -4} \frac{x^3 + 9x^2 + 26x + 24}{x + 4} \]

\textbf{Solution:} Note that this rational function is not defined if $x = -4$. However, $-4$ is a root of both the numerator and the denominator, so we can factor (using either long or synthetic division) and cancel.
\[ \lim_{x \rightarrow -4} \frac{x^3 + 9x^2 + 26x + 24}{x + 4} = \lim_{x \rightarrow -4} \frac{(x + 4)(x^2 + 5x + 6)}{x + 4} = \lim_{x \rightarrow -4} (x^2 + 5x + 6) = 2 \]


  
\vspace{1cm}
\end{enumerate}

\end{document}

\documentclass{article}
\usepackage{neb-titles}
\usepackage{neb-macros}
\usepackage{flexfig}

\begin{document}

\ActivityTitle[class=Calculus I, number=1, name=Limits (Solutions)]

\begin{enumerate}
\item Compute the following limit. \[ \lim_{x \rightarrow 2} \left( 4 x^2 + 7 x + 7 \right) \]

\textbf{Solution:} Since this expression is a polynomial, we can find the limit as $x$ approaches 2 by evaluating at 2.

\begin{eqnarray*}
\lim_{x \rightarrow 2} \left( 4 x^2 + 7 x + 7 \right)
 & = & 4 (2)^2 + 7 (2) + 7 \\
 & = & 16 + 14 + 7 \\
 & = & 37.
\end{eqnarray*}


  
\vspace{1cm}

\item Compute the following limit. \[ \displaystyle\lim_{x \rightarrow 2} \frac{x^2 + 4x - 12}{x - 2} \]

\textbf{Solution:} Note that this rational function is not defined at $x = 2$. However, factoring the numerator we have 

\begin{eqnarray*}
\lim_{x \rightarrow 2} \frac{x^2 + 4x - 12}{x - 2}
 & = & \lim_{x \rightarrow 2} \frac{(x - 2)(x + 6)}{x - 2} \\
 & = & \lim_{x \rightarrow 2} \frac{\cancel{(x - 2)}(x + 6)}{\cancel{x - 2}} \\
 & = & \lim_{x \rightarrow 2} (x + 6) \\
 & = & 8. \\
\end{eqnarray*}


  
\vspace{1cm}

\item Compute the following limit. \[ \lim_{x \rightarrow 7} \frac{x^2 - 49}{x - 7} \]

\textbf{Solution:} Note that this expression is not defined if $x = 7$. However, we can factor the numerator as a difference of squares and cancel.

\[ \lim_{x \rightarrow 7} \frac{x^2 - 49}{x - 7} = \lim_{x \rightarrow 7} \frac{(x - 7)(x + 7)}{x - 7} = \lim_{x \rightarrow 7} (x + 7) = 14 \]


  
\vspace{1cm}

\item Compute the following limit. \[ \lim_{x \rightarrow 4} \frac{x - 4}{\sqrt{x} - 2} \]

\textbf{Solution:} Note that this expression is not defined if $x = 4$. However, we can factor the numerator as a difference of squares.

\[ \lim_{x \rightarrow 4} \frac{x - 4}{\sqrt{x} - 2} = \lim_{x \rightarrow 4} \frac{(\sqrt{x} - 2)(\sqrt{x} + 2)}{\sqrt{x} - 2} = \lim_{x \rightarrow 4} (\sqrt{x} + 2) = 4 \]


  
\vspace{1cm}

\item Compute the following limit. \[ \lim_{x \rightarrow 23} \frac{\sqrt{x - 7} - 4}{x - 23} \]

\textbf{Solution:} Note that this expression is not defined if $x = 23$. But also note that if we multiply the numerator by its radical conjugate, something nice happens:
\[ \left(\sqrt{x - 7} - 4\right)\left(\sqrt{x - 7} + 4\right) = x - 7 - 16 = x - 23. \]
Let's try multiplying by 1, but write 1 as $\sqrt{x - 7} + 4$ over itself.
\begin{eqnarray*}
\lim_{x \rightarrow 23} \frac{\sqrt{x - 7} - 4}{x - 23} & = & \lim_{x \rightarrow 23} \left( \frac{\sqrt{x - 7} - 4}{x - 23} \cdot \frac{\sqrt{x - 7} + 4}{\sqrt{x - 7} + 4} \right) \\
 & = & \lim_{x \rightarrow 23} \frac{x - 23}{(x - 23)(\sqrt{x - 7} + 4)} \\
 & = & \lim_{x \rightarrow 23} \frac{1}{\sqrt{x - 7} + 4} \\
 & = & \frac{1}{8}
\end{eqnarray*}


  
\vspace{1cm}

\item Compute the following limit. \[ \lim_{x \rightarrow 8} \frac{\sqrt{x - 4} - 2}{x - 8} \]

\textbf{Solution:} Note that this expression is not defined if $x = 8$. But also note that if we multiply the numerator by its radical conjugate, something nice happens:
\[ \left(\sqrt{x - 4} - 2\right)\left(\sqrt{x - 4} + 2\right) = x - 4 - 4 = x - 8. \]
Let's try multiplying by 1, but write 1 as $\sqrt{x - 4} + 2$ over itself.
\begin{eqnarray*}
\lim_{x \rightarrow 8} \frac{\sqrt{x - 4} - 2}{x - 8} & = & \lim_{x \rightarrow 8} \left( \frac{\sqrt{x - 4} - 2}{x - 8} \cdot \frac{\sqrt{x - 4} + 2}{\sqrt{x - 4} + 2} \right) \\
 & = & \lim_{x \rightarrow 8} \frac{x - 8}{(x - 8)(\sqrt{x - 4} + 2)} \\
 & = & \lim_{x \rightarrow 8} \frac{1}{\sqrt{x - 4} + 2} \\
 & = & \frac{1}{4}
\end{eqnarray*}


  
\vspace{1cm}

\item Compute the following limit. \[ \lim_{x \rightarrow 3} \frac{x^3 - 3x^2 - x + 3}{x - 3} \]

\textbf{Solution:} Note that this rational function is not defined if $x = 3$. However, $3$ is a root of both the numerator and the denominator, so we can factor (either by grouping or using long or synthetic division) and cancel.
\begin{eqnarray*}
\lim_{x \rightarrow 3} \frac{x^3 - 3x^2 - x + 3}{x - 3} & = & \lim_{x \rightarrow 3} \frac{x^2(x - 3) - 1(x - 3)}{x - 3} \\
 & = & \lim_{x \rightarrow 3} \frac{(x - 3)(x^2 - 1)}{x - 3} \\
 & = & \lim_{x \rightarrow 3} (x^2 - 1) \\
 & = & 8
\end{eqnarray*}


  
\vspace{1cm}

\item Compute the following limit. \[ \lim_{x \rightarrow 1} \frac{x^3 + 4x^2 + x - 6}{x - 1} \]

\textbf{Solution:} Note that this rational function is not defined if $x = 1$. However, $1$ is a root of both the numerator and the denominator, so we can factor (using either long or synthetic division) and cancel.
\[ \lim_{x \rightarrow 1} \frac{x^3 + 4x^2 + x - 6}{x - 1} = \lim_{x \rightarrow 1} \frac{(x - 1)(x^2 + 5x + 6)}{x - 1} = \lim_{x \rightarrow 1} (x^2 + 5x + 6) = 12 \]


  
\vspace{1cm}
\end{enumerate}

\end{document}

\documentclass{article}
\usepackage{neb-titles}
\usepackage{neb-macros}
\usepackage{flexfig}

\begin{document}

\ActivityTitle[class=Calculus I, number=1, name=Limits (Solutions)]

\begin{enumerate}
\item Compute the following limit. \[ \lim_{x \rightarrow 3} \left( 2 x^2 + 6 x + 7 \right) \]

\textbf{Solution:} Since this expression is a polynomial, we can find the limit as $x$ approaches 3 by evaluating at 3.

\begin{eqnarray*}
\lim_{x \rightarrow 3} \left( 2 x^2 + 6 x + 7 \right)
 & = & 2 (3)^2 + 6 (3) + 7 \\
 & = & 18 + 18 + 7 \\
 & = & 43.
\end{eqnarray*}


  
\vspace{1cm}

\item Compute the following limit. \[ \displaystyle\lim_{x \rightarrow 2} \frac{x^2 + 3x - 10}{x - 2} \]

\textbf{Solution:} Note that this rational function is not defined at $x = 2$. However, factoring the numerator we have 

\begin{eqnarray*}
\lim_{x \rightarrow 2} \frac{x^2 + 3x - 10}{x - 2}
 & = & \lim_{x \rightarrow 2} \frac{(x - 2)(x + 5)}{x - 2} \\
 & = & \lim_{x \rightarrow 2} \frac{\cancel{(x - 2)}(x + 5)}{\cancel{x - 2}} \\
 & = & \lim_{x \rightarrow 2} (x + 5) \\
 & = & 7. \\
\end{eqnarray*}


  
\vspace{1cm}

\item Compute the following limit. \[ \lim_{x \rightarrow 6} \frac{x^2 - 36}{x - 6} \]

\textbf{Solution:} Note that this expression is not defined if $x = 6$. However, we can factor the numerator as a difference of squares and cancel.

\[ \lim_{x \rightarrow 6} \frac{x^2 - 36}{x - 6} = \lim_{x \rightarrow 6} \frac{(x - 6)(x + 6)}{x - 6} = \lim_{x \rightarrow 6} (x + 6) = 12 \]


  
\vspace{1cm}

\item Compute the following limit. \[ \lim_{x \rightarrow 16} \frac{x - 16}{\sqrt{x} - 4} \]

\textbf{Solution:} Note that this expression is not defined if $x = 16$. However, we can factor the numerator as a difference of squares.

\[ \lim_{x \rightarrow 16} \frac{x - 16}{\sqrt{x} - 4} = \lim_{x \rightarrow 16} \frac{(\sqrt{x} - 4)(\sqrt{x} + 4)}{\sqrt{x} - 4} = \lim_{x \rightarrow 16} (\sqrt{x} + 4) = 8 \]


  
\vspace{1cm}

\item Compute the following limit. \[ \lim_{x \rightarrow 20} \frac{\sqrt{x - 4} - 4}{x - 20} \]

\textbf{Solution:} Note that this expression is not defined if $x = 20$. But also note that if we multiply the numerator by its radical conjugate, something nice happens:
\[ \left(\sqrt{x - 4} - 4\right)\left(\sqrt{x - 4} + 4\right) = x - 4 - 16 = x - 20. \]
Let's try multiplying by 1, but write 1 as $\sqrt{x - 4} + 4$ over itself.
\begin{eqnarray*}
\lim_{x \rightarrow 20} \frac{\sqrt{x - 4} - 4}{x - 20} & = & \lim_{x \rightarrow 20} \left( \frac{\sqrt{x - 4} - 4}{x - 20} \cdot \frac{\sqrt{x - 4} + 4}{\sqrt{x - 4} + 4} \right) \\
 & = & \lim_{x \rightarrow 20} \frac{x - 20}{(x - 20)(\sqrt{x - 4} + 4)} \\
 & = & \lim_{x \rightarrow 20} \frac{1}{\sqrt{x - 4} + 4} \\
 & = & \frac{1}{8}
\end{eqnarray*}


  
\vspace{1cm}

\item Compute the following limit. \[ \lim_{x \rightarrow 34} \frac{\sqrt{x - 9} - 5}{x - 34} \]

\textbf{Solution:} Note that this expression is not defined if $x = 34$. But also note that if we multiply the numerator by its radical conjugate, something nice happens:
\[ \left(\sqrt{x - 9} - 5\right)\left(\sqrt{x - 9} + 5\right) = x - 9 - 25 = x - 34. \]
Let's try multiplying by 1, but write 1 as $\sqrt{x - 9} + 5$ over itself.
\begin{eqnarray*}
\lim_{x \rightarrow 34} \frac{\sqrt{x - 9} - 5}{x - 34} & = & \lim_{x \rightarrow 34} \left( \frac{\sqrt{x - 9} - 5}{x - 34} \cdot \frac{\sqrt{x - 9} + 5}{\sqrt{x - 9} + 5} \right) \\
 & = & \lim_{x \rightarrow 34} \frac{x - 34}{(x - 34)(\sqrt{x - 9} + 5)} \\
 & = & \lim_{x \rightarrow 34} \frac{1}{\sqrt{x - 9} + 5} \\
 & = & \frac{1}{10}
\end{eqnarray*}


  
\vspace{1cm}

\item Compute the following limit. \[ \lim_{x \rightarrow 5} \frac{x^3 - 5x^2 - 4x + 20}{x - 5} \]

\textbf{Solution:} Note that this rational function is not defined if $x = 5$. However, $5$ is a root of both the numerator and the denominator, so we can factor (either by grouping or using long or synthetic division) and cancel.
\begin{eqnarray*}
\lim_{x \rightarrow 5} \frac{x^3 - 5x^2 - 4x + 20}{x - 5} & = & \lim_{x \rightarrow 5} \frac{x^2(x - 5) - 4(x - 5)}{x - 5} \\
 & = & \lim_{x \rightarrow 5} \frac{(x - 5)(x^2 - 4)}{x - 5} \\
 & = & \lim_{x \rightarrow 5} (x^2 - 4) \\
 & = & 21
\end{eqnarray*}


  
\vspace{1cm}

\item Compute the following limit. \[ \lim_{x \rightarrow 2} \frac{x^3 - 6x^2 + 11x - 6}{x - 2} \]

\textbf{Solution:} Note that this rational function is not defined if $x = 2$. However, $2$ is a root of both the numerator and the denominator, so we can factor (using either long or synthetic division) and cancel.
\[ \lim_{x \rightarrow 2} \frac{x^3 - 6x^2 + 11x - 6}{x - 2} = \lim_{x \rightarrow 2} \frac{(x - 2)(x^2 - 4x + 3)}{x - 2} = \lim_{x \rightarrow 2} (x^2 - 4x + 3) = -1 \]


  
\vspace{1cm}
\end{enumerate}

\end{document}
